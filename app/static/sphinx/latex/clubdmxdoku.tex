%% Generated by Sphinx.
\def\sphinxdocclass{report}
\documentclass[letterpaper,10pt,ngerman]{sphinxmanual}
\ifdefined\pdfpxdimen
   \let\sphinxpxdimen\pdfpxdimen\else\newdimen\sphinxpxdimen
\fi \sphinxpxdimen=.75bp\relax

\PassOptionsToPackage{warn}{textcomp}
\usepackage[utf8]{inputenc}
\ifdefined\DeclareUnicodeCharacter
% support both utf8 and utf8x syntaxes
  \ifdefined\DeclareUnicodeCharacterAsOptional
    \def\sphinxDUC#1{\DeclareUnicodeCharacter{"#1}}
  \else
    \let\sphinxDUC\DeclareUnicodeCharacter
  \fi
  \sphinxDUC{00A0}{\nobreakspace}
  \sphinxDUC{2500}{\sphinxunichar{2500}}
  \sphinxDUC{2502}{\sphinxunichar{2502}}
  \sphinxDUC{2514}{\sphinxunichar{2514}}
  \sphinxDUC{251C}{\sphinxunichar{251C}}
  \sphinxDUC{2572}{\textbackslash}
\fi
\usepackage{cmap}
\usepackage[T1]{fontenc}
\usepackage{amsmath,amssymb,amstext}
\usepackage{babel}



\usepackage{times}
\expandafter\ifx\csname T@LGR\endcsname\relax
\else
% LGR was declared as font encoding
  \substitutefont{LGR}{\rmdefault}{cmr}
  \substitutefont{LGR}{\sfdefault}{cmss}
  \substitutefont{LGR}{\ttdefault}{cmtt}
\fi
\expandafter\ifx\csname T@X2\endcsname\relax
  \expandafter\ifx\csname T@T2A\endcsname\relax
  \else
  % T2A was declared as font encoding
    \substitutefont{T2A}{\rmdefault}{cmr}
    \substitutefont{T2A}{\sfdefault}{cmss}
    \substitutefont{T2A}{\ttdefault}{cmtt}
  \fi
\else
% X2 was declared as font encoding
  \substitutefont{X2}{\rmdefault}{cmr}
  \substitutefont{X2}{\sfdefault}{cmss}
  \substitutefont{X2}{\ttdefault}{cmtt}
\fi


\usepackage[Sonny]{fncychap}
\ChNameVar{\Large\normalfont\sffamily}
\ChTitleVar{\Large\normalfont\sffamily}
\usepackage{sphinx}

\fvset{fontsize=\small}
\usepackage{geometry}


% Include hyperref last.
\usepackage{hyperref}
% Fix anchor placement for figures with captions.
\usepackage{hypcap}% it must be loaded after hyperref.
% Set up styles of URL: it should be placed after hyperref.
\urlstyle{same}

\addto\captionsngerman{\renewcommand{\contentsname}{Inhalt:}}

\usepackage{sphinxmessages}
\setcounter{tocdepth}{1}



\title{ClubDMX Doku}
\date{28.01.2022}
\release{1.0}
\author{Gunther Seiser}
\newcommand{\sphinxlogo}{\vbox{}}
\renewcommand{\releasename}{Release}
\makeindex
\begin{document}

\ifdefined\shorthandoff
  \ifnum\catcode`\=\string=\active\shorthandoff{=}\fi
  \ifnum\catcode`\"=\active\shorthandoff{"}\fi
\fi

\pagestyle{empty}
\sphinxmaketitle
\pagestyle{plain}
\sphinxtableofcontents
\pagestyle{normal}
\phantomsection\label{\detokenize{inhalt::doc}}


Version 1.0


\chapter{README}
\label{\detokenize{readme:readme}}\label{\detokenize{readme::doc}}
ClubDMX ist eine Lichtsteuerung für kleine bis mittelgroße Anwendungsbereiche:
Bars, Probenräume, Ausstellungen, Wohnbereiche. Das Ziel der Entwicklung ist
eine auch für Laien einfach zu bedienende Software.
\begin{itemize}
\item {} 
Läuft auf einem Raspberry PI

\item {} 
Bedienung über Web\sphinxhyphen{}Interface, Midi und OSC.

\item {} 
Verbindung zu den DMX\sphinxhyphen{}Geräten über diverse Hardware
(USB\sphinxhyphen{}Dongles, Artnet, …)

\item {} 
Geräte\sphinxhyphen{}Einzelsteuerung und Lichtszenen

\item {} 
Portable Datenstruktur, alle Daten in CSV\sphinxhyphen{}Tabellen

\item {} 
Angepasst an den Benutzer über Login

\end{itemize}


\section{Online testen}
\label{\detokenize{readme:online-testen}}
Das Web\sphinxhyphen{}Interface läuft zum Testen online. Hier im Testbetrieb kann die
Funktionalität getestet werden. Eine Hardware\sphinxhyphen{}Steuerung ist damit natürlich
nicht möglich. Es sind verschiedene \sphinxstylestrong{Räume} vorbereitet: Eine Bar,
ein Probenraum, ein Wohnzimmer und andere Beispiele. Hier gehts zu
\sphinxhref{https://guntherseiser.pythonanywhere.com/}{ClubDMX online} .


\section{Support}
\label{\detokenize{readme:support}}
Schick mir eine Mail: \sphinxhref{mailto:gunther.seiser.63@gmail.com}{gunther.seiser.63@gmail.com}


\chapter{Features}
\label{\detokenize{features:features}}\label{\detokenize{features::doc}}

\section{Web\sphinxhyphen{}Interface}
\label{\detokenize{features:web-interface}}
ClubDMX läuft auf einem Raspberry PI (oder auf einem anderen Computer) und
wird durch eine Webseite bedient. Auf dem Raspberry läuft also die Software
von ClubDMX und ein Webserver. ClubDMX wird von einem im selben
(lokalen) Netzwerk befindlichen Gerät bedient. Das kann ein Tablet, Samrtphone
oder ein Rechner sein. Es wird dazu keine App benötigt, der Browser genügt.

Die Webseite ist im Responsive Design gestaltet und passt sich an die
Größe des jeweiligen Bildschirms an.


\section{Open Lighting Architecture}
\label{\detokenize{features:open-lighting-architecture}}
Die DMX Verbindung erfolgt über \sphinxhref{https://www.openlighting.org/}{OLA}.
Damit stehen eine Reihe von Hardware\sphinxhyphen{}Optionen zur Verfügung, wie zum Beispiel
Enttec DMX USB Pro, DMXking. OLA ermöglicht auch diverse Ethernet Protokolle
wie ArtNet, sACN und andere. Eine Übersicht über die Möglichkeiten zur
DMX\sphinxhyphen{}Ausgabe findest du \sphinxhref{https://www.openlighting.org/ola/}{hier}.


\section{Szenen}
\label{\detokenize{features:szenen}}
Bei der Entwicklung stand als erstes und grundlegendes Ziel:
\sphinxstylestrong{Lichtszenen (=Cues)} einfach zu erzeugen und zu bedienen. Genau diese
Eigenschaft ist es, die vielen kostengünstigen Lichtsteuerungen
aus meiner Sicht fehlt.
\begin{description}
\item[{Wie sind also hier in ClubDMX die Möglichkeiten für Lichtszenen?}] \leavevmode\begin{itemize}
\item {} 
Cues mit Fader.

\item {} 
Cues mit Button.

\item {} 
Cue mit Top\sphinxhyphen{}Priorität zur Bedienung der einzelnen Geräte und
Erzeugung von weiteren Cues. Siehe {\hyperref[\detokenize{grundlagen:topcuelabel}]{\sphinxcrossref{\DUrole{std,std-ref}{Topcue}}}}

\end{itemize}

\end{description}


\section{Daten in CSV\sphinxhyphen{}Dateien}
\label{\detokenize{features:daten-in-csv-dateien}}
Die Programmdaten werden in CSV\sphinxhyphen{}Dateien gespeichert. Das sind Textdateien,
die mit jedem Editor (Notepad, Excel, Libre Office, …) bearbeitet werden
können. Natürlich ist auch eine Bearbeitung in ClubDMX möglich.
Es stehen Zeilen\sphinxhyphen{} und Zellen\sphinxhyphen{}Bearbeitungsoptionen zur Verfügung.


\section{Angepasst an Benutzer}
\label{\detokenize{features:angepasst-an-benutzer}}
Indem man sich auf der Webseite anmeldet, werden die benutzerspezifischen
Möglichkeiten und Daten aktiv. Ohne Anmeldung ist keine Bedienung der
DMX\sphinxhyphen{}Geäte möglich.


\section{MIDI}
\label{\detokenize{features:midi}}
Außer der Webseite kann ClubDMX auch über MIDI bedient werden. Vorzugsweise über
ein Korg NanoKontrol\sphinxhyphen{}2. Die Buttons und Fader des NanoKontrol2 können mit den
Buttons und Fadern verbunden werden.


\section{OSC Input}
\label{\detokenize{features:osc-input}}
Über \sphinxhref{https://de.wikipedia.org/wiki/Open\_Sound\_Control}{OSC}
kann eine Verbindung von einer externen Software zu ClubDMX hergestellt
werden. Damit können alle Geräte mit allen Attributen, die Szenen\sphinxhyphen{}Fader und
Szenen\sphinxhyphen{}Buttons bedient werden. Getestet mit
\sphinxhref{https://troikatronix.com/}{Isadora} .


\chapter{Erste Schritte}
\label{\detokenize{erste_schritte:erste-schritte}}\label{\detokenize{erste_schritte:erste-schritte-label}}\label{\detokenize{erste_schritte::doc}}
Das ist eine kurze Zusammenfassung der Arbeitsschritte, um ein neu
installiertes ClubDMX zum Leben zu erwecken.
\begin{itemize}
\item {} 
Anmelden

\item {} \begin{description}
\item[{Leeren Raum oder bestehenden Raum laden}] \leavevmode
\sphinxcode{\sphinxupquote{Einrichtung \sphinxhyphen{}\textgreater{} Raum \sphinxhyphen{}\textgreater{} neuen Raum anlegen}}

\end{description}

\item {} 
Patch nach vorhandener Hardware erstellen:
\begin{quote}

Patch  \sphinxstyleemphasis{\_neu} speichern als  \textless{}neuer Patchname\textgreater{}

\sphinxcode{\sphinxupquote{neue Zeile(n)}} erstellt einen oder mehrere neue Heads.
Im Formular die entsprechenden Daten eingeben.

Für weitere Hinweise siehe {\hyperref[\detokenize{patch:neupatchlabel}]{\sphinxcrossref{\DUrole{std,std-ref}{Neuen Patch erstellen}}}}
\end{quote}

\item {} 
Config:
\begin{quote}

Config  \sphinxstyleemphasis{\_neu} speichern als \textless{}neuer Configname\textgreater{}

Falls ClubDMX und OLA am selben Rechner laufen, dann ist die OLA\sphinxhyphen{}Adresse die
Default\sphinxhyphen{}Adresse \sphinxcode{\sphinxupquote{127.0.0.1}}. Falls nicht, dann
in \sphinxcode{\sphinxupquote{Einrichtung \sphinxhyphen{}\textgreater{} Config \sphinxhyphen{}\textgreater{} OLA\sphinxhyphen{}Adresse}} die IP\sphinxhyphen{}Adresse
eintragen. Entsprechend der vorhandenen Hardware die Anzahl der
Universen eintragen.
\end{quote}

\item {} 
OLA:
\begin{quote}

Wenn die OLA IP\sphinxhyphen{}Adresse korrekt ist, dann kann über
\sphinxcode{\sphinxupquote{Einrichtung \sphinxhyphen{}\textgreater{} Config \sphinxhyphen{}\textgreater{} OLA einrichten}}
zur OLA Administration gewechselt werden.

TODO: Seite zur OLA Einrichtung…

Die Verbindung zur Hardware kann in
\sphinxcode{\sphinxupquote{ola.html \sphinxhyphen{}\textgreater{} Universes \sphinxhyphen{}\textgreater{} \textless{}uniname\textgreater{} \sphinxhyphen{}\textgreater{} DMX Console}} getestet werden.
\end{quote}

\item {} 
Stage:
\begin{quote}

Stage \sphinxstyleemphasis{\_neu} speichern als \textless{}neuer Stagename\textgreater{}

\sphinxcode{\sphinxupquote{Import Patch}} legt für jeden in der Stage noch nicht vorhandenen Head
ein Symbol an, mit Headnummer, Name und Kommentar aus Patch.
\end{quote}

\item {} 
Heads verwenden, um Cues, Fader und Buttons zu erzeugen.
\begin{quote}

In \sphinxcode{\sphinxupquote{Stage \sphinxhyphen{}\textgreater{} Single}} können die Geräte einzeln getestet werden.

In \sphinxcode{\sphinxupquote{Info \sphinxhyphen{}\textgreater{} DMX Output}} kann das errechnete DMX\sphinxhyphen{}Signal
angesehen werden.
\end{quote}

\end{itemize}


\chapter{Einrichtung}
\label{\detokenize{einrichten:einrichtung}}\label{\detokenize{einrichten:einrichten-label}}\label{\detokenize{einrichten::doc}}
ClubDMX kommt mit sehr sparsamen Hardware\sphinxhyphen{}Komponenten aus. Als
Basis\sphinxhyphen{}Hardware kann ein Raspberry PI eingesetzt werden, das ist aber
nicht zwingend notwendig. ClubDMX läuft auf jeder Hardware, wo
Python\sphinxhyphen{}3 mit einigen Erweiterungen (die zusätzlich zu Python
installiert werden) einsatzbereit ist.


\section{ClubDMX und OLA}
\label{\detokenize{einrichten:clubdmx-und-ola}}
Open Lighting Architecture (OLA) ist die Schnittstelle zur Hardware,
daher muss auch OLA installiert werden. Zur Einrichtung von OLA
findest du
\sphinxhref{https://www.openlighting.org/ola/getting-started/downloads/}{hier weitere Informationen.}
Zur Installation von OLA auf Mac kann ich leider
nichts beitragen, ich habe OLA auf dem Raspberry und auf (Debian) Linux
installiert und verwendet.

Der große Vorteil von OLA besteht darin, dass damit eine ganze Palette
von Ethernet Protokollen und USB/Serial/Netzwerk Geräten zur
Anbindung zur Verfügung stehen.

Ein einfacher Hardware\sphinxhyphen{}Aufbau mit einem Raspberry PI sieht so aus:

\noindent\sphinxincludegraphics{{basic_setup}.jpg}

ClubDMX und OLA lassen sich sehr einfach auf mehrere Universen erweitern.
Falls ClubDMX in einem größeren Aufbau gedacht ist, können auch mehrere
Raspberry PIs eingesetzt werden. Oder es können an einen Raspberry PI
mehrere USB DMX Adapter angesteckt und für verschiedene Universen
konfiguriert werden.

Ein größerer Hardware\sphinxhyphen{}Aufbau mit mehreren Raspberry PIs könnte so aussehen:

\noindent\sphinxincludegraphics{{ext_setup}.jpg}


\section{Verzeichnisse}
\label{\detokenize{einrichten:verzeichnisse}}
Die Verzeichnisstruktur besteht aus zwei Ordnern, einem Code\sphinxhyphen{}Ordner und
einem Raum\sphinxhyphen{}Ordner. Für diese Ordner gibt es Vorgabe\sphinxhyphen{}Namen, nämlich
\sphinxcode{\sphinxupquote{clubdmx\_code}} und \sphinxcode{\sphinxupquote{clubdmx\_rooms}} .

Über die \sphinxstyleemphasis{Environment}\sphinxhyphen{}Variablen \sphinxcode{\sphinxupquote{CLUBDMX\_CODEPATH}} und \sphinxcode{\sphinxupquote{CLUBDMX\_ROOMPATH}}
können auch andere Verzeichnisse festgelegt werden.


\section{Raum einrichten}
\label{\detokenize{einrichten:raum-einrichten}}\label{\detokenize{einrichten:roomsetup}}
Die Einrichtung eines Raumes erzeugt im Raum\sphinxhyphen{}Verzeichnis (\sphinxcode{\sphinxupquote{clubdmx\_rooms}})
einen neuen Ordner. Hier werden alle projektspezifischen Daten gespeichert:
Patch, Cues, Fader, Buttons, Stage.

Es gibt einige Möglichkeiten zur Erzeugung eines neuen Raums.
\begin{quote}

Einen neuen Raum anlegen,

einen bestehenden Raum unter einem anderen Namen sichern,

einen auf USB\sphinxhyphen{}Stick gepeicherten Raum laden (restore),

einen gezippten Raum hochladen.
\end{quote}

\noindent\sphinxincludegraphics{{setup_room_page}.jpg}


\section{Config}
\label{\detokenize{einrichten:config}}\label{\detokenize{einrichten:configsetup}}
In einer Config sind die im Einsatz befindlichen Hardware\sphinxhyphen{}Komponenten, das
Fader\sphinxhyphen{} und Button\sphinxhyphen{}Layout, das Stage\sphinxhyphen{}Layout und weitere Einstellungen
gespeichert. In einem Raum können auch mehrere Configs vorhanden sein, zum
Beispiel um unterschiedliche Theaterprojekte zu speichern. Der Zusammenhang
von Raum und Config wird in
{\hyperref[\detokenize{grundlagen:raum-config-label}]{\sphinxcrossref{\DUrole{std,std-ref}{Räume und Config}}}} noch näher erläutert.

Die Config ist in verschiedene Bereiche gegliedert. Diese Bereiche sind in
Tabs zusammengefasst. Der erste Bereich definiert die Hardware\sphinxhyphen{}nahen
Einstellungen: Auswahl eines
{\hyperref[\detokenize{patch:patchlabel}]{\sphinxcrossref{\DUrole{std,std-ref}{Patch}}}}, OLA und OSC.

\noindent\sphinxincludegraphics{{setup_config_page}.jpg}

Der nächste Config\sphinxhyphen{}Berich behandelt die Bedienelemente. Hier werden die
Tabellen ausgewählt, die {\hyperref[\detokenize{einrichten:stage}]{\sphinxcrossref{Stage}}}, {\hyperref[\detokenize{einrichten:fader}]{\sphinxcrossref{Fader}}} und {\hyperref[\detokenize{einrichten:buttons}]{\sphinxcrossref{Buttons}}} definieren.

Das Aussehen der Webseite kann hier durch eine Auswahl eines Themas
verändert werden.

\noindent\sphinxincludegraphics{{setup_elements_page}.jpg}

Der nächste Bereich beinhaltet die Midi\sphinxhyphen{}Einstellungen. Das ist zwar auch
ein Hardware\sphinxhyphen{}nahes Thema, bekommt aber aus Gründen der Übersichtlichkeit
einen eigenen Tab.

\noindent\sphinxincludegraphics{{setup_midi_page}.jpg}

Der nächste Bereich in der Config gibt einen Überblick über die
verwendeten Datenbank\sphinxhyphen{}Tabellen. In ClubDMX sind ja alle Daten in
CSV\sphinxhyphen{}Tabellen gespeichert. Diese Tabellen sind im Raum\sphinxhyphen{}Ordner gepeichert.
Hier können sie angesehen und editiert werden. Dazu stehen Zeilen\sphinxhyphen{} und
Zellen\sphinxhyphen{}Funktionen zur Verfügung.

Mit Copy und Paste können einzelne oder mehrere Zeilen von einer Tabelle
in eine andere übertragen werden, Zeilen können gelöscht oder neue Zeilen
können über Eingabeformular angelegt werden.

Beim Editieren von einzelnen Zellen wird geprüft, ob der eingegebene Wert
in die Zelle „passt“, das heißt, ob die Regeln für diese Zelle erfüllt
sind.

Als weitere Hilfsmittel finden sich auf diesem Tab Funktionen zur
Bereinigung der Datenbank: unbenutzet Cues löschen, alle Änderungen sichern
oder alle Änderungen speichern.

\noindent\sphinxincludegraphics{{setup_data_page}.jpg}

Für den User \sphinxstyleemphasis{Admin} gibt es einen weiteren Tab, auf dem einige Tools zu
finden sind: Backup und Restore der User\sphinxhyphen{}Datenbank, einen Link zur
OLA\sphinxhyphen{}einrichtung und diverse Debug\sphinxhyphen{}Tools.


\section{Stage einrichten}
\label{\detokenize{einrichten:stage-einrichten}}\label{\detokenize{einrichten:stage}}
Sobald ein {\hyperref[\detokenize{patch:patchlabel}]{\sphinxcrossref{\DUrole{std,std-ref}{Patch}}}} angelegt ist, kann dieser in die
Stage importiert werden. Die Heads werden in einem Standard\sphinxhyphen{}Raster eingefügt.
Im Modus \sphinxstyleemphasis{SELECT} können die Heads in Position und Größe verändert
werden.

\noindent\sphinxincludegraphics{{stage_page}.jpg}


\section{Fader einrichten}
\label{\detokenize{einrichten:fader-einrichten}}\label{\detokenize{einrichten:fader}}
Die Fader befinden sich in zwei Tabellen. Eine Tabelle beinhaltet die Fader, die
auf der Exec\sphinxhyphen{}Seite sichtbar sind.
Eine zweite Tabelle beinhaltet die Fader, die
nur auf der Fader\sphinxhyphen{}Seite zu finden sind. Das hat den Zweck, für den
Basic\sphinxhyphen{}Benutzer nur die für ihn relevanten Fader auf seiner Start\sphinxhyphen{}Seite
platzieren zu können.

\noindent\sphinxincludegraphics{{fader_page}.jpg}

Durch Drücken auf das Zahnrad\sphinxhyphen{}Symbol gelangst du zur Einrichtungs\sphinxhyphen{}Seite für die
Fader. Hier kannst du mit den üblichen Tabellen\sphinxhyphen{}Funktionen (copy/paste,
Feld editieren, sichern als, neue Zeile) die Fader einrichten. Die beiden oben
genannten Tabellen (Exec\sphinxhyphen{}Fader, Zusatz\sphinxhyphen{}Fader) stehen hier zur Bearbeitung.
Mit cut/paste können die Fader von einer Tabelle in die andere
verschoben werden.

\noindent\sphinxincludegraphics{{fader_setup_page}.jpg}


\section{Buttons einrichten}
\label{\detokenize{einrichten:buttons-einrichten}}\label{\detokenize{einrichten:buttons}}
Die Buttons befinden sich in drei Tabellen:  Exec\sphinxhyphen{}Buttons oben, Exec\sphinxhyphen{}Buttons
unten und zusätzliche Buttons. Je nachdem, in welcher Tabelle die Buttons
platziert werden, wird die Position der Buttons festgelegt.

\noindent\sphinxincludegraphics{{button_page}.jpg}

Über das Zahnrad\sphinxhyphen{}Symbol kann auch hier zur Einrichtungs\sphinxhyphen{}Seite
gewechselt werden. Auch hier kann über die üblichen Tabellen\sphinxhyphen{}Funktionen
die Anordnung, die Beschriftung und die Funktion der Buttons eingerichtet
werden.

Mit Copy/Paste können Fader zu Buttons gemacht werden und Buttons zu Fadern.
Einfach Zeile(n) ausschneiden und in die gewünschte Tabelle einfügen.

\noindent\sphinxincludegraphics{{button_setup_page}.jpg}


\section{Exec\sphinxhyphen{}Seite einrichten}
\label{\detokenize{einrichten:exec-seite-einrichten}}\label{\detokenize{einrichten:execsetup}}
Die Einrichtung der Fader und Buttons bestimmt das Erscheinungsbild
der Exec\sphinxhyphen{}Seite. Es gibt daher keine eigene Einrichtungs\sphinxhyphen{}Seite für die
Exec\sphinxhyphen{}Seite, siehe also {\hyperref[\detokenize{einrichten:fader}]{\sphinxcrossref{Fader}}} und   {\hyperref[\detokenize{einrichten:buttons}]{\sphinxcrossref{Buttons}}} .

\noindent\sphinxincludegraphics{{exec_page}.jpg}

Weitere Informationen zur Einrichtung findest du im Kapitel
{\hyperref[\detokenize{navigation:navigation-label}]{\sphinxcrossref{\DUrole{std,std-ref}{ClubDMX benützen}}}}, da gibt es Details zu den einzelnen Tabs.


\section{Einrichten von zeitabhängiger Steuerung mit Crontab (in Linux)}
\label{\detokenize{einrichten:einrichten-von-zeitabhangiger-steuerung-mit-crontab-in-linux}}\label{\detokenize{einrichten:crontab-label}}
Zur Unterstützung von Fixinstallationen mit Tages\sphinxhyphen{} oder
Wochen\sphinxhyphen{}abhängigen Anforderungen gibt es die Möglichkeit, über den
Linux\sphinxhyphen{}Befehl \sphinxcode{\sphinxupquote{crontab}} Lichtszenen zu steuern. Hier ist exemplarisch
skizziert, wie das funktioniert.

Über OSC können externe Programme auf ClubDMX zugreifen. Diese Eigenschaft
nützt das Kommandozeilen\sphinxhyphen{}Tool \sphinxcode{\sphinxupquote{sendosc.sh}}
(im Programm\sphinxhyphen{}Verzeichnis zu finden).

Beispiel für einen Crontab\sphinxhyphen{}Eintrag, der jede Minute aufgerufen wird und
den dritten Button aus der oberen Exec\sphinxhyphen{}Zeile betätigt:

\begin{sphinxVerbatim}[commandchars=\\\{\}]
\PYG{o}{*} \PYG{o}{*} \PYG{o}{*} \PYG{o}{*} \PYG{o}{*} \PYG{o}{/}\PYG{n}{home}\PYG{o}{/}\PYG{n}{pi}\PYG{o}{/}\PYG{n}{clubdmx\PYGZus{}code}\PYG{o}{/}\PYG{n}{sendosc}\PYG{o}{.}\PYG{n}{sh} \PYG{o}{\PYGZhy{}}\PYG{o}{\PYGZhy{}}\PYG{n}{address}\PYG{o}{=}\PYG{o}{/}\PYG{n}{exebutton1}\PYG{o}{/}\PYG{l+m+mi}{3}
\end{sphinxVerbatim}

Die unterstützten OSC\sphinxhyphen{}Befehle finden sich auf der ClubDMX Webseite und auch
hier in der Doku im Kapitel {\hyperref[\detokenize{navigation:navigation-label}]{\sphinxcrossref{\DUrole{std,std-ref}{ClubDMX benützen}}}}.

Mit dem Linux\sphinxhyphen{}Befehl \sphinxcode{\sphinxupquote{crontab \sphinxhyphen{}e}} wird die Crontab editiert, mit dem
Befehl \sphinxcode{\sphinxupquote{crontab \sphinxhyphen{}l}} wird sie angezeigt und mit \sphinxcode{\sphinxupquote{crontab \sphinxhyphen{}r}} wird sie
gelöscht.


\chapter{ClubDMX benützen}
\label{\detokenize{navigation:clubdmx-benutzen}}\label{\detokenize{navigation:navigation-label}}\label{\detokenize{navigation::doc}}
Über die Startseite (erreichbar über das Symbol \sphinxincludegraphics{{favicon-16x16}.png}) können die
wichtigsten Module aufgerufen werden, über die Navigationsleiste
können alle Module aufgerufen werden.

Hier ist eine Übersicht über die Navigation:


\section{Stage}
\label{\detokenize{navigation:stage}}\begin{itemize}
\item {} 
Stage

\item {} 
Stage kompakt

\item {} 
Single

\end{itemize}

Die erste Wahl in der Navigation ist die \sphinxstylestrong{Stage}. Dieses Modul besteht aus
mehreren Webseiten. Die erste Wahl ist die Stage mit beweglichen und
selektierbaren Elementen. Als Idee dahinter ist, die \sphinxstylestrong{Geräte (= Heads)}
wie in einem Grundriss eines Raumplanes zu positionieren. Damit können
die Geräte auf dem Bildschirm so angeordnet werden, wie sie sich im reellen
Raum befinden.

\noindent\sphinxincludegraphics{{stage_page}.jpg}

Da bei Samrtphones und anderen kleinen Anzeigen der Platz oft nicht
ausreicht, wird eine zweite Stage\sphinxhyphen{}Ansicht zur Verfügung gestellt:
\sphinxstylestrong{Stage kompakt}. Hier werden die Elemente neben\sphinxhyphen{} und untereinander so
platziert, dass der Bildschirm optimal genutzt wird (Responsive Design).

\noindent\sphinxincludegraphics{{stage_kompakt}.jpg}

In der \sphinxstylestrong{Single} Seite kann mittels Navigationspfeilen \sphinxcode{\sphinxupquote{\textless{}\textless{}}},
\sphinxcode{\sphinxupquote{\textless{}}}, \sphinxcode{\sphinxupquote{\textgreater{}}}, \sphinxcode{\sphinxupquote{\textgreater{}\textgreater{}}} jedes Gerät
einzeln in den {\hyperref[\detokenize{grundlagen:topcuelabel}]{\sphinxcrossref{\DUrole{std,std-ref}{Topcue}}}} geladen werden.
Für das ausgewählte Gerät werden im
Arbeitsbereich Fader angezeigt, die die einzelnen \sphinxstyleemphasis{Attribute}
bedienbar machen.

\noindent\sphinxincludegraphics{{single_page}.jpg}


\section{Exec}
\label{\detokenize{navigation:exec}}
Die Exec\sphinxhyphen{}Seite enthält Buttons und Fader. Sie kann so konfiguriert werden,
dass sie einen Button\sphinxhyphen{}Bereich, einen Fader\sphinxhyphen{}Bereich und einen weiteren
Button\sphinxhyphen{}Bereich anzeigt. Jeder dieser Bereiche kann auch leer sein.

Das Ziel der Exec\sphinxhyphen{}Seite ist ein schneller und simpler Zugang zu Szenen.
Auf die Editiermöglichkeiten wurde daher verzichtet. Zum
Einrichten und Editieren der Exec\sphinxhyphen{}Seite wechselt man auf die Fader\sphinxhyphen{} oder
Button\sphinxhyphen{}Seite. Genaueres dazu findest du unter {\hyperref[\detokenize{einrichten:fader}]{\sphinxcrossref{\DUrole{std,std-ref}{Fader einrichten}}}} und
{\hyperref[\detokenize{einrichten:buttons}]{\sphinxcrossref{\DUrole{std,std-ref}{Buttons einrichten}}}} .

\noindent\sphinxincludegraphics{{exec_page}.jpg}


\section{Fader}
\label{\detokenize{navigation:fader}}
Die Fader\sphinxhyphen{}Seite war der Startpunkt für die Arbeiten zu ClubDMX, diese Seite
war das erste Ziel: Lichtszenen (Cues) über einen
Webseiten\sphinxhyphen{}Fader zu bedienen.

Die dazu nötigen Voraussetzungen waren dann doch eher umfangreich, aber die
für die Anwender wichtige Klarheit ist hier nach wie vor gegeben: Ein Fader für
jeden Cue,
ein Text zur Identifizierung, ein Button zur Ansicht, welche Geräte mit diesem
Fader bedient werden und ein Button zum Editieren des Cues.

Neue Fader hinzufügen oder entfernen, die Reihenfolge der Fader ändern: Das
alles ist möglich, und zwar in der Fadertabelle.
Siehe {\hyperref[\detokenize{einrichten:fader}]{\sphinxcrossref{\DUrole{std,std-ref}{Fader einrichten}}}}

In der Fadertabelle ist es auch möglich, einen MIDI\sphinxhyphen{}Regler einem Fader
zuzuorden.


\section{Button}
\label{\detokenize{navigation:button}}
Ganz gleich wie die Fader\sphinxhyphen{}Seite ist die Button\sphinxhyphen{}Seite für die einfache
Bedienung konzipiert: Hier können Szenen (Cues) über die Webseite ein\sphinxhyphen{} oder
ausgeschalten werden.

Buttons unterscheiden sich von Fadern dadurch, dass bei
Buttons nur die Endpunkte 0 und 100\% (Aus oder Ein) vom Benutzer gewählt
werden können.

Der Übergang von Aus zu Ein beziehungsweise von Ein zu Aus kann mit
Fade\sphinxhyphen{}Zeiten verknüpft werden. Dann bestimmt die Zeit in Sekunden einen linearen
Übergang von Ursprungs\sphinxhyphen{} zum Zielzustand.

Es gibt drei verschiedene Arten von Buttons:
\begin{itemize}
\item {} 
Schalter

\item {} 
Taster

\item {} 
Auswahlschalter

\end{itemize}

Ein Schalter verändert mit jedem Drücken seinen Status, von \sphinxstyleemphasis{aus} nach \sphinxstyleemphasis{ein}
und wieder zurück.
Ein Taster ist nur dann \sphinxstyleemphasis{ein}, wenn er gedrückt und gehalten wird. Das
funktioniert nur auf einem Midi\sphinxhyphen{}Keyboard, nicht auf der Webseite.

Ein Auswahlschalter hängt mit anderen Auswahlschaltern zusammen: Mehrere
Auswahlschalter bilden eine Gruppe (mit einer Gruppen\sphinxhyphen{}Nummer). Wird ein
Schalter aus dieser Gruppe \sphinxstyleemphasis{ein}\sphinxhyphen{}geschalten, dann wird damit ein momentan
aktiver Schalter aus dieser Gruppe \sphinxstyleemphasis{aus}\sphinxhyphen{}geschalten. Es kann also immer
nur ein Schalter aus einer Gruppe \sphinxstyleemphasis{ein} sein.

Es können beliebig viele Buttons eingerichtet werden. Auswahlschalter
müssen nicht nebeneinander platziert wrden, sie werden durch die Gruppen\sphinxhyphen{}Nummer
als zusammengehörig definiert.

Auf der Button\sphinxhyphen{}Seite sind die für den Benutzer essentiellen Informationen
enthalten: Text zur Identifizierung, Status (Aus: grauer Rand,
Ein: roter Rand), weiters Buttons zur Ansicht der Cue\sphinxhyphen{}Informationen und zum
Editieren.

In der Buttontabelle werden alle Infos zu den Buttons verwaltet. Hier können
Buttons hinzugefügt, entfernt und in der Reihenfolge verändert werden.
Weiters werden hier die Fade\sphinxhyphen{}Zeiten und die Zuordnung zu MIDI\sphinxhyphen{}Tasten
eigetragen.


\section{Einrichtung}
\label{\detokenize{navigation:einrichtung}}
Die Einrichtung von ClubDMX ist schon recht vielseitig, meine Hoffnung ist,
dass sie trotzdem noch übersichtlich ist, um sich schnell zurecht zu finden.
\begin{itemize}
\item {} 
Raum

\item {} 
Config

\item {} 
Bedienelemente

\item {} 
Midi

\item {} 
Datenbank

\item {} 
Admin

\end{itemize}

Die Einrichtung ist in verschiedene Bereiche gegliedert, die sich auf der
Webseite in Tabs wiederfinden.

Im \sphinxstylestrong{Raum}\sphinxhyphen{}Tab werden die ersten und elementaren Einrichtungen getätigt:
Einen neuen Raum anlegen, den Raum wechseln, den aktiven Raum umbenennen.
Im Raum\sphinxhyphen{}Tab befinden sich auch die Optionen zum Backup und Restore.

Die nächste Ebene der Einrichtung ist die \sphinxstylestrong{Config}, deren Hardware\sphinxhyphen{}Komponenten
sich im Config\sphinxhyphen{}Tab befinden. Das
Kernstück der Konfiguration ist der \sphinxstylestrong{Patch}. Die zweite wesentliche
Hardware\sphinxhyphen{}Komponente ist OLA, das die pyhsische Zuordnung über DMX zu den
Geräten herstellt.

Falls  für einen Raum
mehrere Configs angelegt wurden, kann hier eine davon ausgewählt werden. Die
aktuelle Config kann auch unter einem anderen Namen gespeichert werden.

Zur Erläuterung der Begriffe Raum und Config siehe {\hyperref[\detokenize{grundlagen:raum-config-label}]{\sphinxcrossref{\DUrole{std,std-ref}{Räume und Config}}}} .

Nach dem Öffnen einer Config werden der
{\hyperref[\detokenize{patch:patchlabel}]{\sphinxcrossref{\DUrole{std,std-ref}{Patch}}}} ausgewählt, die OLA\sphinxhyphen{}Adresse und die Anzahl der Universen
eingestellt.

Im \sphinxstylestrong{Config}\sphinxhyphen{}Tab wird auch OSC Input ein\sphinxhyphen{} oder ausgeschalten.

Im \sphinxstylestrong{Bedienelemente}\sphinxhyphen{}Tab sind die Tabellen auswählbar, die die zur Nutzung
relevanten Seiten definieren: Stage, Fader und Buttons. Damit können für einen
Anwendungsbereich die nötige Lokalisierung der Geräte auf der Stage und die
passenden Fader und Buttons bereitgestellt werden.

Für den Fall, dass ClubDMX als Steuerung für eine Fix\sphinxhyphen{}Installation verwendet
wird, die nach einem Ausschalten mit den letzten Einstellungen wieder
hochgefahren werden soll, gibt es den Schalter \sphinxstyleemphasis{Levels speichern}.

Verschiedene Stil\sphinxhyphen{}Themen für die Webseite können im Bedienelemente\sphinxhyphen{}Tab
ausgewählt werden.

Im \sphinxstylestrong{Midi}\sphinxhyphen{}Tab können maximal vier Midi\sphinxhyphen{}Geräte als Input definiert werden. Diese
Midi\sphinxhyphen{}Geräte können Fadern und Buttons zugeordnet werden. Damit können Cues
auch ohne Website bedient werden.

Der \sphinxstylestrong{Datenbank}\sphinxhyphen{}Tab ermöglicht einen Zugriff auf die Tabellen, die hier
bearbeitet werden können. Je nach gewähltem {\hyperref[\detokenize{grundlagen:bearbeitungsmoduslabel}]{\sphinxcrossref{\DUrole{std,std-ref}{Bearbeitungsmodus}}}}
können hier
Zell\sphinxhyphen{} oder Zeilenbearbeitungen gemacht werden.

Umfassende Erläuterungen zum Thema Einrichtung finden sich im
Kapitel {\hyperref[\detokenize{einrichten:einrichten-label}]{\sphinxcrossref{\DUrole{std,std-ref}{Einrichtung}}}} .


\section{Info}
\label{\detokenize{navigation:info}}
Information zur Software und den aktuellen DMX\sphinxhyphen{}Output
\begin{itemize}
\item {} 
DMX\sphinxhyphen{}Output: zeigt alle DMX\sphinxhyphen{}Werte, die nicht Null sind. Das ist
zur Kontrolle gedacht, da man hier sieht, welche Berechnungen ClubDMX
für die einzelnen Fader, Buttons und den Topcue vorgenommen hat.

\item {} 
Info: Kurz\sphinxhyphen{}Informationen

\item {} 
Doku: Umfangreiche Information zu verschiedenen Themen: Erste Schritte,
Einrichtung, die einzelnen Navigations\sphinxhyphen{}Punkte (Module) und ein Bereich
zu Grundlagen der Lichtsteuerung.

\end{itemize}


\section{Benutzer\sphinxhyphen{}Datenbank}
\label{\detokenize{navigation:benutzer-datenbank}}
Die Benützung von ClubDMX ist nur mit einem Login möglich.
Es gibt verschiedene Rollen der Berechtigung, zur Zeit sind das
Basic, Standard und Admin.

Die genaue Kategorisierung, welche Aktionen mit welcher Rolle
durchgeführt werden dürfen, ist noch in Arbeit. Den aktuellen Stand zu den
Benutzer\sphinxhyphen{}Rollen siehe {\hyperref[\detokenize{benutzer:benutzer-label}]{\sphinxcrossref{\DUrole{std,std-ref}{Benutzer}}}} .


\section{OSC Input}
\label{\detokenize{navigation:osc-input}}
Dieses Modul ermöglicht die Verbindung mit externen Programmen wie zum Beispiel
\sphinxhref{https://troikatronix.com/}{Isadora} . Über
\sphinxhref{https://de.wikipedia.org/wiki/Open\_Sound\_Control}{OSC} können
folgende Aktionen in ClubDMX ausgeführt werden:
\begin{itemize}
\item {} 
\sphinxcode{\sphinxupquote{/head \textless{}attribut\textgreater{} \textless{}headnr\textgreater{} \textless{}level\textgreater{}}} ermöglicht de Zugriff auf sämtliche
Geräte und deren Attribute.

\begin{DUlineblock}{0em}
\item[] \sphinxcode{\sphinxupquote{\textless{}attribut\textgreater{}}} bezeichnet den Attribut\sphinxhyphen{}Namen, so wie er in der Head\sphinxhyphen{}Definition
\item[] angelegt ist.
\item[] \sphinxcode{\sphinxupquote{\textless{}headnr\textgreater{}}} ist eine Ganzzahl entsprechend der Headnummer im Patch.
\item[] \sphinxcode{\sphinxupquote{\textless{}level\textgreater{}}} ist eine Gleitzahl im Bereich zwischen 0 und 1.
\end{DUlineblock}

\item {} 
\sphinxcode{\sphinxupquote{/clear}} leert den {\hyperref[\detokenize{grundlagen:topcuelabel}]{\sphinxcrossref{\DUrole{std,std-ref}{Topcue}}}} .

\item {} 
\sphinxcode{\sphinxupquote{/fader/\textless{}nummer\textgreater{} \textless{}level\textgreater{}}} ändert den Level eines Faders.

\item {} 
\sphinxcode{\sphinxupquote{/exefader/\textless{}nummer\textgreater{} \textless{}level\textgreater{}}} ändert den Level eines Faders aus der
Exec\sphinxhyphen{}Seite.

\begin{DUlineblock}{0em}
\item[] \sphinxcode{\sphinxupquote{\textless{}nummer\textgreater{}}} ist eine Ganzzahl entsprechend der Fadertabelle.
\item[] \sphinxcode{\sphinxupquote{\textless{}level\textgreater{}}} ist eine Gleitzahl im Bereich zwischen 0 und 1.
\end{DUlineblock}

\item {} 
\sphinxcode{\sphinxupquote{/button/\textless{}nummer\textgreater{}}} betätigt einen Button.

\item {} 
\sphinxcode{\sphinxupquote{/exebutton1/\textless{}nummer\textgreater{}}} betätigt einen Button aus der oberen Reihe
der Exec\sphinxhyphen{}Seite.

\item {} 
\sphinxcode{\sphinxupquote{/exebutton2/\textless{}nummer\textgreater{}}} betätigt einen Button aus der unteren Reihe
der Exec\sphinxhyphen{}Seite.

\sphinxcode{\sphinxupquote{\textless{}nummer\textgreater{}}} ist eine Ganzzahl entsprechend der Buttontabelle.

\end{itemize}


\chapter{Benutzer}
\label{\detokenize{benutzer:benutzer}}\label{\detokenize{benutzer:benutzer-label}}\label{\detokenize{benutzer::doc}}
ClubDMX ist mit einer Benutzerstruktur ausgestattet. Je nach Rolle des
Benutzers sind Funktionen und Seiteninhalte verfügbar.

Es gibt drei Rollen: Basic, Standard und Admin.


\section{Basic}
\label{\detokenize{benutzer:basic}}
Benutzer mit dieser Rolle können Stage, Fader und Buttons bedienen. Damit
kann ein Basic\sphinxhyphen{}Benutzer alle voreingestellten Szenen aufrufen und auf
alle Geräte zugreifen. Eine Veränderung von Szenen ist nicht möglich.


\section{Standard}
\label{\detokenize{benutzer:standard}}
Benutzer mit Standard\sphinxhyphen{}Berechtigung haben vollen Zugriff auf die Programmierung,
sie können Veränderungen nach ihren Vorstellungen vornehmen. Alle
Änderungen werden so gespeichert, dass sie auch rückgängig gemacht
werden können. Das Rückgängig\sphinxhyphen{}Machen kann vom Standard\sphinxhyphen{}Benutzer oder vom
Admin gemacht werden.

Ein Standard\sphinxhyphen{}Benutzer kann auch neue Benutzer anlegen, denen er die Rolle
Standard oder Basic zuordnen kann.


\section{Admin}
\label{\detokenize{benutzer:admin}}
Gegenüber dem Standard\sphinxhyphen{}Benutzer hat der Admin Rechte in der
Form, dass er einen aktuellen Programmstatus in Teilen oder zur Gänze
permanent speichern kann. Permanent gespeicherte Daten können dann nicht
mehr rückgängig gemacht werden.

Der Admin kann auch weitere Benutzer mit Admin\sphinxhyphen{}Rechten anlegen, Benutzer
löschen und die Passwörter von Benutzern zurücksetzen.


\section{Funktionen}
\label{\detokenize{benutzer:funktionen}}
Die verschiedenen Seiten zur Benutzerverwaltung sind entsprechend der Rolle des
jeweiligen Benutzers verfügbar. Die komplette Benutzerverwaltung ist nur
Admins zugänglich, ein Standard\sphinxhyphen{}Benutzer sieht die Liste der Benutzer und
kann neue Benutzer anlegen. Ein Basic\sphinxhyphen{}Benutzer kann sich nur an\sphinxhyphen{} und abmelden
und das Passwort ändern.


\chapter{Patch}
\label{\detokenize{patch:patch}}\label{\detokenize{patch:patchlabel}}\label{\detokenize{patch::doc}}
Jedes physisch vorhandene Gerät wird im Patch eingetragen.
Damit wird es für ClubDMX verfügbar.

Aufruf über die Navigation: \sphinxcode{\sphinxupquote{Einrichtung \sphinxhyphen{}\textgreater{} Config \sphinxhyphen{}\textgreater{} Patch bearbeiten}}.

Zu den einzelnen Spalten:
\begin{itemize}
\item {} 
HeadNr: Jedes Gerät wird mit einer Zahl identifiziert.
Es können auch mehrere Geräte dieselbe HeadNr haben.
Sie werden dann von ClubDMX gleich behandelt.

\item {} 
HeadType: Jedem Gerät wird eine Head\sphinxhyphen{}Datei zugeordnet.
Das ist die eindeutige Beschreibung der einzelnen DMX\sphinxhyphen{}Kanäle
für das zu patchende Gerät.

\item {} 
Addr: Die DMX\sphinxhyphen{}Startadresse. Setzt sich aus zwei Komponenten zusammen,
die mit Bindestrich getrennt sind: \sphinxstyleemphasis{\textless{}Universum\textgreater{}\sphinxhyphen{}\textless{}Startadresse\textgreater{}}.
Von ClubDMX wird beim Erstellen von neuen Zeilen geprüft,
ob die Adressen korrekt im Bereich von 1\sphinxhyphen{}512 liegen.
Nicht geprüft wird, ob eventuell Überschneidungen mit anderen Heads
auftreten.

\item {} 
Name: Das ist ein Textfeld, das für den Benutzer zur
Identifizierung des Hardware\sphinxhyphen{}Gerätes dient. Beim Import des Patch
in die Stage wird der Name zur Identifikation des Stage\sphinxhyphen{}Elementes
verwendet.

\item {} 
Gel (Farbe): Das ist ein Textfeld, das für die Farbnummer eines
konventionellen Scheinwerfers gedacht ist. Dient zur Dokumentation eines
Projektes und ist zur Zeit nicht weiter ausgeführt.

\item {} 
Comment: Wie schon der Name sagt, für Kommentar und Anmerkung

\end{itemize}


\section{Neuen Patch erstellen}
\label{\detokenize{patch:neuen-patch-erstellen}}\label{\detokenize{patch:neupatchlabel}}
In jeder Datenbank\sphinxhyphen{}Kategorie gibt es eine Tabelle mit Namen  \sphinxstyleemphasis{\_neu},
die als Prototyp fungiert. So auch in der Kategorie Patch.
Sichere diesen Prototyp unter einem neuen Namen und beginne
nun mit der Erstellung der Zeilen.

Mit dem Button \sphinxcode{\sphinxupquote{Neue Zeile(n)}} wird ein Formular zum Erzeugen
eines neuen Eintrags geöffnet. Fülle das Formular zur
Erstellung eines neuen Eintrags aus.

Mit dem Formular können mehrere Heads vom selben Head\sphinxhyphen{}Typ in einem Schritt
erzeugt werden. Je nach Head\sphinxhyphen{}Typ werden die Adressen für die einzelnen Heads
berechnet, mit dem Abstand, der sich aus der Head\sphinxhyphen{}Definition errechnet.

Dabei wird auch der maximale DMX\sphinxhyphen{}Wert von 512 berücksichtigt.
Unter Umständen können daher weniger als die gewünschten Heads erzeugt werden,
wenn ein oder mehrere Attribute einen DMX\sphinxhyphen{}Wert über 512 hätten.

Im Bearbeitungsmodus SELECT kannst du nun sehr schnell weitere
Zeilen mit Copy/Paste erzeugen, die du im Modus EDIT entsprechend anpasst.

\noindent\sphinxincludegraphics{{patch_new}.jpg}


\chapter{Grundlagen}
\label{\detokenize{grundlagen:grundlagen}}\label{\detokenize{grundlagen::doc}}

\section{DMX}
\label{\detokenize{grundlagen:dmx}}
Über das DMX\sphinxhyphen{}Protokoll können alle möglichen unterschiedlichen Geräte bedient
werden, zum Beispiel Dimmer, LED\sphinxhyphen{}Scheinwerfer, Nebelmaschinen etc. Diese Geräte
haben ein oder mehrere Attribute, wie Intensität, Farben,
Position oder Geschwindigkeit.
Ein Gerät hat eine Startadresse, das ist eine Zahl zwischen 1 und 512.
So ein Zahlenbereich von 1\sphinxhyphen{}512 wird als Universum bezeichnet, was natürlich
verglichen mit dem realen Universum reichlich übertrieben ist.
Daher ist es mit dieser Lichttechniker\sphinxhyphen{}Definition von Universum möglich,
dass sich in einem Raum mehrere Universen befinden.


\section{Hardware}
\label{\detokenize{grundlagen:hardware}}
Geräte\sphinxhyphen{}Eigenschaften werden mit Attributen beschrieben.
Diese Beschreibung nennt man Head\sphinxhyphen{}Definition.
In ClubDMX werden Head\sphinxhyphen{}Definitionen wie alle anderen Daten in CSV\sphinxhyphen{}Tabellen
gespeichert.

Ein Gerät im Universum könnte zum Beispiel ein LED\sphinxhyphen{}Scheinwerfer mit den
Farben rot, grün und blau sein. Diese Farben sind die Attribute, die unabhängig
voneinander in der Intensität verändert werden können.
In einer simplen Lichtsteuerung erhält jede Farbe einen Regler zugewiesen,
mit dem dann die Intensität eingestellt wird.
Wenn ich jetzt aus den drei Farben einen schönen Farbton gemischt habe,
dann wäre es wünschenswert, wenn ich diesen Farbton nun in der Intensität
verändern könnte. Das ist mit drei Reglern etwas schwierig, da ich ja
die drei Regler im richtigen Verhältnis zueinander auf oder ab schieben soll.

Und hier kommt ein Konzept der Lichttechnik ins Spiel, das sich
in den letzten Jahren verbreitet hat, nämlich der \sphinxstyleemphasis{virtuelle Dimmer}.
Ein virtueller Dimmer ist ein Faktor (zwischen 0 und 1),
mit dem jeder Farbwert des LED\sphinxhyphen{}Scheinwerfers multipliziert wird.
In der Lichttechnik ist der virtuelle Dimmer nun ebenso ein Regler,
der zusätzlich zu den Farben des LED\sphinxhyphen{}Scheinwerfers zur Verfügung steht.


\section{Räume und Config}
\label{\detokenize{grundlagen:raume-und-config}}\label{\detokenize{grundlagen:raum-config-label}}
In ClubDMX werden in einem \sphinxstyleemphasis{Raum} alle Daten der Datenbank zusammengefasst, die
für die Steuerung notwendig sind. Das sind unter anderem Konfigurationen,
Patches, Head\sphinxhyphen{}Daten, Cues, Stages, Fadertabellen und Buttontabellen.
Durch diese Struktur ist es einfach, alle zu einem Projekt gehörenden Daten
zu sichern und auch, ClubDMX in verschiedenen Anwendungsbereichen zu betreiben.

Die Aufgabe der Config ist es, für einen Raum die Anforderungen an eine spezielle
Aufgabe anzupassen, zum Beispiel für eine Aufführung oder eine
Personengruppe, die einen Veranstaltungsort nützt.

Besipiele dazu wären, wenn man für einen Club verschiedene Configs für
Standard\sphinxhyphen{} und Session\sphinxhyphen{}Bespielung oder für einen Probenraum verschiedene
Configs für die jeweiligen Nutzer anlegt.


\section{Bearbeitungsmodus}
\label{\detokenize{grundlagen:bearbeitungsmodus}}\label{\detokenize{grundlagen:bearbeitungsmoduslabel}}
Wie aus anderen Tabellenkalkulationen bekannt, werden zwei unterschiedliche
Editiermethoden benötigt: Zellenweises Editieren und Zeilenoprerationen.

In ClubDMX ist es genauso, daher wird der Begriff des Bearbeitungsmodus
eingeführt. Das Editieren von Text in Zellen geschieht im Modus \sphinxstyleemphasis{EDIT},
Zeilenoperationen wie Ausschneiden, Kopieren und Einfügen geschieht im
Modus \sphinxstyleemphasis{SELECT}.

Auf der \sphinxstylestrong{Stage}\sphinxhyphen{}Seite hat der Modus \sphinxstyleemphasis{SELECT} eine weitere Aufgabe:
In diesem Modus werden Symbole selektiert,
die dann in der Größe verändert und verschoben
werden können. Für die selektierten \sphinxstyleemphasis{Heads} werden Attribut\sphinxhyphen{}Slider
angezeigt, mit denen die Attribute verändert werden können.


\section{HTP und LTP}
\label{\detokenize{grundlagen:htp-und-ltp}}\label{\detokenize{grundlagen:htpltplabel}}
Zu Zeiten, als in der Theaterbeleuchtung außer Dimmern bestenfalls noch
ansteuerbare Schalter zu finden waren, gab es eine klare Regelung: Wenn
ein Dimmer von einem Regler in einem bestimmten Wert angefordert wurde
und von einem anderen Regler in einem anderen Wert, so galt der höhere Wert.

In Zeiten von Moving Lights, Farbwechslern und Equipment, das andere
Parameter als Intensität kennt, ist diese einfache Regel nicht mehr ausreichend.

Ein einfaches Beispiel: Wird die Position eines Moving Light mit zwei Werten
für die
x\sphinxhyphen{} und die y\sphinxhyphen{}Achse beschrieben, so definiert zum Beispiel Szene 1 die Position
als 50\% für x und 50\% für y. Wenn nun in einer nachfolgenden Szene 10\% für x und
90\% für y angefordert wird, dann eräbe sich nach der „höchsten\sphinxhyphen{}Wert\sphinxhyphen{}Regel“ als
Kombination der beiden Szenen ein x\sphinxhyphen{}Wert von 50\% und ein y\sphinxhyphen{}Wert von 90\%. Das
entspricht dann keiner der beiden Szenen und ist als Ergebnis völlig
unbrauchbar.

Daher braucht es für solche Attribute wie Position eine andere Regel. Diese
ist ebenso simpel wie die Regel „der höchste Wert gilt“, die Regel lautet:
Der letzte Wert gilt.

In der englischsprachigen Welt gibt es natürlich auch Überlegungen
diesbezüglich, und als Resultat davon haben sich bei uns auch die
englischen Abkürzungen für diese Regeln zur Beschreibung durchgesetzt:
HTP und LTP.

HTP bedeutet \sphinxstylestrong{highest takes precedence}, LTP steht für
\sphinxstylestrong{latest takes precedence}.

Als Zusammenfassung dieser beiden Regeln kann für die zeitgemäße Lichtsteuerung
gesagt werden, dass die Intensität nach der HTP\sphinxhyphen{}Regel und alle anderen
Attribute nach der LTP\sphinxhyphen{}Regel gesteuert werden.

Die Ausnahme von dieser Aussage
bilden die RGB\sphinxhyphen{}Farben eines LED\sphinxhyphen{}Scheinwerfers, bei dem die Farben rot, grün und
blau entweder als eigenständige Intensitäten und somit gemäß HTP\sphinxhyphen{}Regel oder als
kombinierter Farbwert und damit gemäß LTP\sphinxhyphen{}Regel behandelt werden. Wie RGB\sphinxhyphen{}Werte
gesteuert werden, hängt von den Präferenzen des Lichttechnikers ab und die
beste Methode ist von der jeweiligen Situation abhängig.

Siehe: \sphinxurl{http://www.thedmxwiki.com/dmx\_basics/ltp\_and\_htp}


\section{Topcue}
\label{\detokenize{grundlagen:topcue}}\label{\detokenize{grundlagen:topcuelabel}}
Hinter dem Begriff \sphinxstylestrong{Topcue} steht eine simple Idee: Das ist der Speicher
der Werte, die neu programmiert oder verändert wurden. Diese Werte sollen
aktuell gültig sein und es soll auch eine Speicherung in einem neuen Cue oder
eine Änderung eines vorhandenen Cues ermöglichen.

Der Topcue ist eine Lichtszene (\sphinxstylestrong{Cue}), die aber nicht gemäß HTP\sphinxhyphen{}Regel
behandelt wird. Für diese Szene gilt, dass jeder hier enthaltene Wert
Vorrang (= Top\sphinxhyphen{}Priorität) hat. Somit kann ein aus der Summe der aktiven
Szenen errechneter Wert außer Kraft und durch einen anderen Wert ersetzt
werden.

Es gibt verschiedene Möglichkeiten, damit Werte in den Topcue gelangen:
\begin{itemize}
\item {} 
Attribute werden im Single\sphinxhyphen{}Modus verändert.

\item {} 
In der Stage werden Geräte ausgewählt und es werden Attribute
verändert.

\item {} 
Ein Cue wird mit dem \sphinxstyleemphasis{Topcue\sphinxhyphen{}Button} in den Topcue aufgenommen.

\end{itemize}

\noindent\sphinxincludegraphics{{topcue1}.jpg}

Sind Werte im Topcue enthalten, dann erscheint unterhalb der
Standard\sphinxhyphen{}Navigationsleiste
eine zweite Navigationsleiste mit den Optionen \sphinxstylestrong{leeren, anzeigen und speichern}.

Mit \sphinxstyleemphasis{leeren} wird der Topcue gelöscht und es gelten wieder die errechneten Werte
für alle Geräte.

Mit \sphinxstyleemphasis{anzeigen} wird der Inhalt des Topcue in einer Liste dargestellt.

Der Button \sphinxstyleemphasis{speichern} eröffnet weitere Möglichkeiten: Die Werte des Topcue
können als neuer Cue, als Fader oder als Button abgespeichert werden.

\noindent\sphinxincludegraphics{{topcue2}.jpg}


\chapter{Raspberry einrichten}
\label{\detokenize{raspberry:raspberry-einrichten}}\label{\detokenize{raspberry:raspberry-label}}\label{\detokenize{raspberry::doc}}
Der Raspberry PI ist eine für ClubDMX getestete Hardware, hier ist
eine Anleitung zur Neu\sphinxhyphen{}Installation von Raspbian Buster als Betriebssystem,
Installation von OLA und Installation von ClubDMX.


\section{Raspberry Buster neu installieren}
\label{\detokenize{raspberry:raspberry-buster-neu-installieren}}\label{\detokenize{raspberry:buster}}\begin{itemize}
\item {} 
Programm Imager von \sphinxurl{https://www.raspberrypi.org/downloads/}

\item {} 
Image erzeugen

\item {} 
Raspi mit Bidschirm und Tatatur starten, anschließend Guide ausführen

\item {} 
Raspi\sphinxhyphen{}Config:

\end{itemize}

\begin{sphinxVerbatim}[commandchars=\\\{\}]
\PYG{n}{sudo} \PYG{n}{raspi}\PYG{o}{\PYGZhy{}}\PYG{n}{config}
\PYG{l+m+mi}{2} \PYG{n}{Network} \PYG{n}{Options}
\PYG{c+c1}{\PYGZsh{} \PYGZhy{}\PYGZgt{} N1 Hostname: Pi\PYGZhy{}Name eintragen (optional)}

\PYG{l+m+mi}{3} \PYG{n}{Boot} \PYG{n}{Options}
\PYG{c+c1}{\PYGZsh{} \PYGZhy{}\PYGZgt{} B2 Wait for Network on Boot}
\PYG{c+c1}{\PYGZsh{} (sonst ist OLA nicht mit allen Plugins ausgestattet)}

\PYG{l+m+mi}{4} \PYG{n}{Localisation} \PYG{n}{Options}
\PYG{c+c1}{\PYGZsh{} \PYGZhy{}\PYGZgt{} Ländercode + Utf\PYGZhy{}8, auch Ländercode für Wlan (wichtig)}

\PYG{l+m+mi}{5} \PYG{n}{Interfacing} \PYG{n}{Options}
\PYG{c+c1}{\PYGZsh{} \PYGZhy{}\PYGZgt{} P2 SSH enable (wichtig)}

\PYG{l+m+mi}{5} \PYG{n}{Interfacing} \PYG{n}{Options}
\PYG{c+c1}{\PYGZsh{} \PYGZhy{}\PYGZgt{} P3 VNC enable (optional)}
\end{sphinxVerbatim}


\section{OLA installieren}
\label{\detokenize{raspberry:ola-installieren}}
Für die Installation von OLA gibt es zwei Möglichkeiten, eine einfache, die das
nicht tagesaktuelle Repository installiert, und eine langwierigere, die das
aktuelle OLA von GIT installiert. Die zweite Version war für mich notwendig in
einem Fall, wo ich als DMX\sphinxhyphen{}Adapter den \sphinxstyleemphasis{Eurolite DMX512 Pro MK2} verwenden
wollte, der war im Repository noch nicht verfügbar. Hier sind beide
Varianten beschrieben, nur eine davon muss installiert werden.

\sphinxstylestrong{Einfache OLA Installation}

\sphinxcode{\sphinxupquote{sudo nano /etc/apt/sources.list}}

hier eintragen:

\begin{sphinxVerbatim}[commandchars=\\\{\}]
\PYG{c+c1}{\PYGZsh{}ola:}
\PYG{n}{deb} \PYG{n}{http}\PYG{p}{:}\PYG{o}{/}\PYG{o}{/}\PYG{n}{apt}\PYG{o}{.}\PYG{n}{openlighting}\PYG{o}{.}\PYG{n}{org}\PYG{o}{/}\PYG{n}{raspbian} \PYG{n}{wheezy} \PYG{n}{main}
\end{sphinxVerbatim}

anschließend Reboot.
Nach dem Neustart:

\sphinxcode{\sphinxupquote{sudo apt\sphinxhyphen{}get install ola}}

Damit ist OLA installiert und startet automatisch bei jedem Neustart.

\sphinxstylestrong{OLA von GIT installieren}

Wenn die \sphinxstyleemphasis{einfache OLA\sphinxhyphen{}Installation} gewählt wurde, kann der folgende Abschnitt
übersprungen werden. Es geht dann weiter im Abschnitt \sphinxstyleemphasis{OLA für ClubDMX vorbereiten}.

Hier sind meine Notizen, die ich während der Installation gemacht habe. Diese
sind auch in
\sphinxhref{https://groups.google.com/g/open-lighting/c/rDIbzhqnWxQ}{Google Groups} nachzulesen.

\begin{sphinxVerbatim}[commandchars=\\\{\}]
\PYG{n}{sudo} \PYG{n}{apt}\PYG{o}{\PYGZhy{}}\PYG{n}{get} \PYG{n}{install} \PYG{n}{git}
\PYG{n}{git} \PYG{n}{clone} \PYG{n}{https}\PYG{p}{:}\PYG{o}{/}\PYG{o}{/}\PYG{n}{github}\PYG{o}{.}\PYG{n}{com}\PYG{o}{/}\PYG{n}{OpenLightingProject}\PYG{o}{/}\PYG{n}{ola}\PYG{o}{.}\PYG{n}{git} \PYG{n}{ola}
\PYG{n}{cd} \PYG{n}{ola}

\PYG{c+c1}{\PYGZsh{} (Error: could not resolve host..., therfore:)}
\PYG{n}{git} \PYG{n}{config} \PYG{o}{\PYGZhy{}}\PYG{o}{\PYGZhy{}}\PYG{k}{global} \PYG{o}{\PYGZhy{}}\PYG{o}{\PYGZhy{}}\PYG{n}{unset} \PYG{n}{http}\PYG{o}{.}\PYG{n}{proxy}
\PYG{n}{git} \PYG{n}{config} \PYG{o}{\PYGZhy{}}\PYG{o}{\PYGZhy{}}\PYG{k}{global} \PYG{o}{\PYGZhy{}}\PYG{o}{\PYGZhy{}}\PYG{n}{unset} \PYG{n}{http}\PYG{o}{.}\PYG{n}{proxy}

\PYG{n}{sudo} \PYG{n}{apt}\PYG{o}{\PYGZhy{}}\PYG{n}{get} \PYG{n}{install} \PYG{n}{autoconf} \PYG{n}{libtool} \PYG{n}{bison} \PYG{n}{flex} \PYG{n}{uuid}\PYG{o}{\PYGZhy{}}\PYG{n}{dev} \PYG{n}{libcppunit}\PYG{o}{\PYGZhy{}}\PYG{n}{dev}
\PYG{n}{sudo} \PYG{n}{apt}\PYG{o}{\PYGZhy{}}\PYG{n}{get} \PYG{n}{install} \PYG{n}{libmicrohttpd}\PYG{o}{\PYGZhy{}}\PYG{n}{dev} \PYG{n}{protobuf}\PYG{o}{\PYGZhy{}}\PYG{n}{compiler} \PYG{n}{libprotobuf}\PYG{o}{\PYGZhy{}}\PYG{n}{dev} \PYG{n}{python}\PYG{o}{\PYGZhy{}}\PYG{n}{protobuf}
\PYG{n}{sudo} \PYG{n}{apt}\PYG{o}{\PYGZhy{}}\PYG{n}{get} \PYG{n}{install} \PYG{n}{libftdi}\PYG{o}{\PYGZhy{}}\PYG{n}{dev} \PYG{n}{liblo}\PYG{o}{\PYGZhy{}}\PYG{n}{dev} \PYG{n}{libavahi}\PYG{o}{\PYGZhy{}}\PYG{n}{client}\PYG{o}{\PYGZhy{}}\PYG{n}{dev} \PYG{n}{libprotoc}\PYG{o}{\PYGZhy{}}\PYG{n}{dev}
\PYG{n}{sudo} \PYG{n}{apt}\PYG{o}{\PYGZhy{}}\PYG{n}{get} \PYG{n}{install} \PYG{n}{libusb}\PYG{o}{\PYGZhy{}}\PYG{l+m+mf}{1.0}\PYG{o}{.}\PYG{l+m+mi}{0}\PYG{o}{\PYGZhy{}}\PYG{n}{dev} \PYG{n}{libcurses5}\PYG{o}{\PYGZhy{}}\PYG{n}{dev} \PYG{n}{pkg}\PYG{o}{\PYGZhy{}}\PYG{n}{config} \PYG{n}{liblo}
\PYG{n}{autoreconf} \PYG{o}{\PYGZhy{}}\PYG{n}{i}
\PYG{o}{.}\PYG{o}{/}\PYG{n}{configure}
\PYG{c+c1}{\PYGZsh{} optionally: ./configure \PYGZhy{}\PYGZhy{}enable\PYGZhy{}rdm\PYGZhy{}tests \PYGZhy{}\PYGZhy{}enable\PYGZhy{}python\PYGZhy{}libs}
\PYG{n}{make}
\PYG{n}{make} \PYG{n}{check}
\PYG{n}{sudo} \PYG{n}{make} \PYG{n}{install}
\PYG{n}{sudo} \PYG{n}{ldconfig}
\end{sphinxVerbatim}

Damit ist die neueste Version von OLA installiert. Nun muss noch Autostart
konfiguriert werden:

\sphinxcode{\sphinxupquote{sudo nano /etc/rc.local}}

hier eintragen vor der letzten Zeile (\sphinxstyleemphasis{exit 0}):

\sphinxcode{\sphinxupquote{su pi \sphinxhyphen{}c "olad \sphinxhyphen{}f"}}

\sphinxstylestrong{Eurolite  usb\sphinxhyphen{}dmx mk2}

Dieser Abschnitt muss nur dann ausgeführt werden, wenn der Eurolite DMX\sphinxhyphen{}Adapter
verwendet werden soll. Diese Anleitung funktioniert mit der neuesten Version
von OLA, nicht aber mit der einfachen OLA\sphinxhyphen{}Installation.
Ansonsten weiter im
Abschnitt \sphinxstyleemphasis{OLA für ClubDMX vorbereiten}.

\sphinxcode{\sphinxupquote{sudo nano /etc/modprobe.d/eurolite\sphinxhyphen{}dmx.conf}}

hier eintragen:

\sphinxcode{\sphinxupquote{blacklist cdc\_acm}}

Eurolite Adapter anstecken und Vendor\sphinxhyphen{}ID und Product\sphinxhyphen{}ID prüfen:

\begin{sphinxVerbatim}[commandchars=\\\{\}]
\PYG{n}{lsusb}
\PYG{n}{sudo} \PYG{n}{nano} \PYG{o}{/}\PYG{n}{etc}\PYG{o}{/}\PYG{n}{udev}\PYG{o}{/}\PYG{n}{rules}\PYG{o}{.}\PYG{n}{d}\PYG{o}{/}\PYG{l+m+mi}{02}\PYG{o}{\PYGZhy{}}\PYG{n}{eurolite}\PYG{o}{\PYGZhy{}}\PYG{n}{dmx}\PYG{o}{.}\PYG{n}{rules}\PYG{p}{:}
\PYG{c+c1}{\PYGZsh{} (this is one line:)}
\PYG{n}{SUBSYSTEM}\PYG{o}{==}\PYG{l+s+s2}{\PYGZdq{}}\PYG{l+s+s2}{usb|usb\PYGZus{}device}\PYG{l+s+s2}{\PYGZdq{}}\PYG{p}{,} \PYG{n}{ACTION}\PYG{o}{==}\PYG{l+s+s2}{\PYGZdq{}}\PYG{l+s+s2}{add}\PYG{l+s+s2}{\PYGZdq{}}\PYG{p}{,} \PYG{n}{ATTRS}\PYG{p}{\PYGZob{}}\PYG{n}{idVendor}\PYG{p}{\PYGZcb{}}\PYG{o}{==}\PYG{l+s+s2}{\PYGZdq{}}\PYG{l+s+s2}{0403}\PYG{l+s+s2}{\PYGZdq{}}\PYG{p}{,}
 \PYG{n}{ATTRS}\PYG{p}{\PYGZob{}}\PYG{n}{idProduct}\PYG{p}{\PYGZcb{}}\PYG{o}{==}\PYG{l+s+s2}{\PYGZdq{}}\PYG{l+s+s2}{6001}\PYG{l+s+s2}{\PYGZdq{}}\PYG{p}{,} \PYG{n}{GROUP}\PYG{o}{=}\PYG{l+s+s2}{\PYGZdq{}}\PYG{l+s+s2}{plugdev}\PYG{l+s+s2}{\PYGZdq{}} \PYG{n}{MODE}\PYG{o}{=}\PYG{l+s+s2}{\PYGZdq{}}\PYG{l+s+s2}{660}\PYG{l+s+s2}{\PYGZdq{}}
\end{sphinxVerbatim}

Nun müssen noch einige conf\sphinxhyphen{}Dateien von OLA angepasst werden. Diese können sich
an unterschiedlichen Orten befinden, im Raspberry PI mit der beschriebenen
Installation aber wahrscheinlich in \sphinxcode{\sphinxupquote{/home/pi/.ola/}}.
Zur Sicherheit am Besten im Browser die
OLA\sphinxhyphen{}Admin Seite 127.0.0.1:9090 öffnen und im Abschnitt \sphinxstyleemphasis{Plugins} eine Seite
öffnnen, dann findet sich hier die \sphinxstyleemphasis{Config Location}.

In den drei Dateien
\sphinxcode{\sphinxupquote{/home/pi/.ola/ola\sphinxhyphen{}opendmx.conf ,
/home/pi/.ola/ola\sphinxhyphen{}usbserial.conf und
/home/pi/.ola/ola\sphinxhyphen{}stageprofi.conf}} jeweils die Zeile
\sphinxcode{\sphinxupquote{enabled = true}} auf \sphinxcode{\sphinxupquote{enabled = false}} ändern.

In der Datei \sphinxcode{\sphinxupquote{ola\sphinxhyphen{}usbdmx.conf}} die Zeile \sphinxcode{\sphinxupquote{enable\_eurolite\_mk2 = true}}
einfügen.

Damit ist nach einem Neustart der Eurolite Adapter verfügbar.


\section{OLA für ClubDMX vorbereiten}
\label{\detokenize{raspberry:ola-fur-clubdmx-vorbereiten}}
Weiter im Browser, auf Seite \sphinxstyleemphasis{127.0.0.1:9090}.

ClubDMX kommuniziert mit OLA über OSC, daher muss in OLA in jedem Universum
ein OSC\sphinxhyphen{}Input eingetragen werden.

Ein oder mehrere Universen anlegen:

\begin{sphinxVerbatim}[commandchars=\\\{\}]
\PYG{n}{Universes} \PYG{o}{\PYGZhy{}}\PYG{o}{\PYGZgt{}} \PYG{n}{Add} \PYG{n}{Universe}
\PYG{n}{Universe} \PYG{n}{ID}\PYG{p}{:} \PYG{l+m+mi}{1}
\PYG{n}{Universe} \PYG{n}{Name}\PYG{p}{:} \PYG{n}{Uni1}
\PYG{n}{Checkbox} \PYG{n}{anhaken} \PYG{n}{bei} \PYG{n}{erster} \PYG{n}{Zeile} \PYG{n}{OSC} \PYG{n}{Device} \PYG{n}{Input} \PYG{o}{/}\PYG{n}{dmx}\PYG{o}{/}\PYG{n}{universe}\PYG{o}{/}\PYG{o}{\PYGZpc{}}\PYG{n}{d}
\PYG{n}{Checkbox} \PYG{n}{für} \PYG{n}{gewünschten} \PYG{n}{Output} \PYG{n}{anhaken}\PYG{o}{.}
\end{sphinxVerbatim}

Für weitere Universen wiederholen.


\section{ClubDMX installieren}
\label{\detokenize{raspberry:clubdmx-installieren}}
Ich entwickle ClubDMX auf einem Windows\sphinxhyphen{}Rechner, daher kann es sein, dass
Shell\sphinxhyphen{}Skripte erst ins Unix\sphinxhyphen{}Format umgewandelt werden müssen. Dazu gibt es
ein Hilfsprogramm namens \sphinxcode{\sphinxupquote{dos2unix}}. Siehe auch
\sphinxhref{https://www.digitalmasters.info/de/das-zeilenende-unter-linux-windows-und-os-x-im-griff/}{hier}

\sphinxcode{\sphinxupquote{sudo apt\sphinxhyphen{}get install dos2unix}}

Und nun zur eigentlichen Installation von ClubDMX. Die aktuelle Version ist
in meinem \sphinxhref{https://drive.google.com/drive/u/0/folders/1obYMWAk5R5nDciTfRPQ6BhFDnAgCs9Bd}{Google Drive}
zu finden.

\sphinxstylestrong{Standard\sphinxhyphen{}Installation}

In der Standard\sphinxhyphen{}Installation werden ein Code\sphinxhyphen{}Verzeichnis und ein Raum\sphinxhyphen{}Verzeichnis
im /home Verzeichnis des Users pi angelegt. Diese Verzeichnisse können auch an
andere orte verschoben werden. Falls das gewünscht wird, dann bitte weiter unten
in den Anmerkungen nachlesen.

Die ZIP\sphinxhyphen{}Datei wird mit Filezilla, WinSCP oder von einem USB\sphinxhyphen{}Stick
ins /home Verzeichnis kopiert,  anschließend mit dem
Befehl

\sphinxcode{\sphinxupquote{unzip clubdmx\_code.zip}}

entpackt. Damit wird das code\sphinxhyphen{}Verzeichnis  \sphinxcode{\sphinxupquote{clubdmx\_code}}
angelegt und mit den aktuellen
Code\sphinxhyphen{}Daten befüllt. Das Raum\sphinxhyphen{}Verzeichnis wird mit dem Befehl

\sphinxcode{\sphinxupquote{mkdir clubdmx\_rooms}}

angelegt.

Beispiele für Räume können von meiner Webseite
\sphinxurl{guntherseiser.pythonanywhere.com} heruntergeladen und in das Raumverzeichnis
übertragen werden.

\sphinxstylestrong{Alias anlegen}

Diese Zeile am Ende von \sphinxcode{\sphinxupquote{/home/pi/.bashrc}} anfügen:

\sphinxcode{\sphinxupquote{alias clubdmx=\textquotesingle{}/home/pi/clubdmx\_code/app\_start.sh\textquotesingle{}}}

\sphinxstylestrong{Python Module}

Alle nötigen Module installieren:

\begin{sphinxVerbatim}[commandchars=\\\{\}]
\PYG{n}{cd} \PYG{n}{clubdmx\PYGZus{}code}
\PYG{o}{.}\PYG{o}{/}\PYG{n}{python\PYGZus{}setup}\PYG{o}{.}\PYG{n}{sh}
\end{sphinxVerbatim}

\sphinxstylestrong{Secret Key}

User\sphinxhyphen{}Datenbank und Cookies funktionieren nur, wenn für die Webseite ein
\sphinxstyleemphasis{SECRET KEY} angelegt wird. Nur um das einmal klarzustellen: Ich gehe davon aus,
dass du mit Cookies einverstanden bist, wenn du mein Programm verwendest. Die
Cookies dienen nur zur Funktion der Webseite und werden nicht für irgendwelche
anderen Zwecke verwendet.

\begin{sphinxVerbatim}[commandchars=\\\{\}]
\PYG{n}{cd} \PYG{n}{clubdmx\PYGZus{}code}
\PYG{n}{nano} \PYG{o}{.}\PYG{n}{env}
\PYG{c+c1}{\PYGZsh{} hier eintragen:}
\PYG{n}{SECRET\PYGZus{}KEY} \PYG{o}{=} \PYG{l+s+sa}{b}\PYG{l+s+s1}{\PYGZsq{}}\PYG{l+s+s1}{84nrf97vzih47vzkd98747}\PYG{l+s+s1}{\PYGZsq{}}
\PYG{c+c1}{\PYGZsh{} nicht genau diesen String, sondern eine zufällige Zeichenkette,}
\PYG{c+c1}{\PYGZsh{} beginnend mit b\PYGZsq{} und abgeschlossen mit \PYGZsq{}}
\end{sphinxVerbatim}

\sphinxstylestrong{ClubDMX starten}

Die letzten Schritte sind schnell erledigt:

\begin{sphinxVerbatim}[commandchars=\\\{\}]
\PYG{n}{cd} \PYG{o}{/}\PYG{n}{home}\PYG{o}{/}\PYG{n}{pi}\PYG{o}{/}\PYG{n}{clubdmx\PYGZus{}code}
\PYG{n}{dos2unix} \PYG{o}{*}\PYG{o}{.}\PYG{n}{sh}
\PYG{n}{chmod} \PYG{o}{+}\PYG{n}{x} \PYG{o}{*}\PYG{o}{.}\PYG{n}{sh}
\PYG{o}{.}\PYG{o}{/}\PYG{n}{app\PYGZus{}start}\PYG{o}{.}\PYG{n}{sh} \PYG{n}{start}
\end{sphinxVerbatim}

\sphinxstylestrong{Autostart einrichten}

Damit ist die ClubDMX installiert. Nun muss noch Autostart
konfiguriert werden. Hier ist eine Unterscheidung zu treffen, je nachdem
der Raspberry in der Konsole  (wird in der Regel für ClubDMX die richtige
Wahl sein) oder mit grafischer Benutzeroberfläche startet (auch dafür wird es
Gründe geben).

\sphinxstyleemphasis{Konsolen\sphinxhyphen{}Start:}

\sphinxcode{\sphinxupquote{sudo nano /etc/rc.local}}

hier eintragen vor der letzten Zeile (\sphinxstyleemphasis{exit 0}):

\begin{sphinxVerbatim}[commandchars=\\\{\}]
\PYG{c+c1}{\PYGZsh{} clubdmx app:}
\PYG{n}{su} \PYG{n}{pi} \PYG{o}{\PYGZhy{}}\PYG{n}{c} \PYG{l+s+s2}{\PYGZdq{}}\PYG{l+s+s2}{/home/pi/clubdmx\PYGZus{}code/app\PYGZus{}start.sh start}\PYG{l+s+s2}{\PYGZdq{}} \PYG{o}{\PYGZam{}}
\end{sphinxVerbatim}

\sphinxstyleemphasis{Desktop\sphinxhyphen{}Start}

Wenn der Pi im Desktop\sphinxhyphen{}Modus laufen soll, dann muss ClubDMX anders gestartet
werden. Daher den Start von ClubDMX nicht in \sphinxcode{\sphinxupquote{/etc/rc.local}} machen, sondern so:

\begin{sphinxVerbatim}[commandchars=\\\{\}]
\PYG{n}{sudo} \PYG{n}{nano} \PYG{o}{/}\PYG{n}{etc}\PYG{o}{/}\PYG{n}{xdg}\PYG{o}{/}\PYG{n}{lxsession}\PYG{o}{/}\PYG{n}{LXDE}\PYG{o}{\PYGZhy{}}\PYG{n}{pi}\PYG{o}{/}\PYG{n}{autostart}
\PYG{c+c1}{\PYGZsh{} hier eintragen vor der letzten Zeile (= @screensaver…):}
\PYG{o}{@}\PYG{o}{/}\PYG{n}{home}\PYG{o}{/}\PYG{n}{pi}\PYG{o}{/}\PYG{n}{clubdmx\PYGZus{}code}\PYG{o}{/}\PYG{n}{app\PYGZus{}start}\PYG{o}{.}\PYG{n}{sh} \PYG{n}{start}
\end{sphinxVerbatim}


\bigskip\hrule\bigskip


\sphinxstylestrong{Anmerkungen}

Die Verzeichnisse für Code und Räume können beliebig positioniert werden.
ClubDMX findet die Verzeichnisse über Environment\sphinxhyphen{}Variablen.

\begin{sphinxVerbatim}[commandchars=\\\{\}]
\PYG{n}{CLUBDMX\PYGZus{}CODEPATH}
\PYG{c+c1}{\PYGZsh{} (default, falls nicht gesetzt: /home/pi/clubdmx\PYGZus{}code)}
\PYG{n}{CLUBDMX\PYGZus{}ROOMPATH}
\PYG{c+c1}{\PYGZsh{} (default, falls nicht gesetzt: /home/pi/clubdmx\PYGZus{}rooms)}
\end{sphinxVerbatim}

Anmerkung: Am Raspberry werden Environment\sphinxhyphen{}Variablen in der Datei
\sphinxcode{\sphinxupquote{/etc/environment}}  eingetragen, zum Beispiel

\sphinxcode{\sphinxupquote{export CLUBDMX\_ROOMPATH=”/home/pi/Documents/rooms”}}


\section{Troubles?}
\label{\detokenize{raspberry:troubles}}
Die Installation ist \sphinxhyphen{} beginnend bei einer leeren SD\sphinxhyphen{}Karte \sphinxhyphen{} nun doch ein
wenig lang. Daher wird vielleicht der eine oder andere Fehler auftauchen.
In diesem Abschnitt der Doku werde ich versuchen, Hilfestellung zur
Fehlersuche zu liefern.

Meine Methode zur Fehlersuche ist, einzelne Abschnitte zu testen, um nach
Möglichkeit einen Fehler in einem bestimmten Bereich zu lokalisieren.
Die erste große Unterteilung ist zwischen Hardware und Software,
die nächste Trennung ist zwischen OLA und ClubDMX.

\sphinxstylestrong{Hardware}
\begin{itemize}
\item {} 
Netzwerk prüfen

Wird der Raspberry mit Bildschirm und Tastatur verwendet oder über das
Netzwerk (WinSCP, Filezilla, ssh)?

IP Adresse des PI. Am Terminal oder der Tastatur eingeben:

\sphinxcode{\sphinxupquote{ip address}}

Eine gültige IP Adresse kann so ermittelt werden. Diese ist wichtig, um
die OLA\sphinxhyphen{}Seite und die Seite von ClubDMX zu finden. Die IP Adresse des Raspberry PI
wird im folgenden immer mit \textless{}PI\sphinxhyphen{}IP\textgreater{} bezeichnet.

\end{itemize}

\sphinxstylestrong{OLA prüfen}

Ich gehe davon aus, dass du am Raspberry mit Bildschirm und Tastatur sitzt.
Wenn nicht, dann bitte zuerst das Netzwerk prüfen.
\begin{itemize}
\item {} 
Ist die OLA Webseite erreichbar?
Im Browser die Seite \textless{}PI\sphinxhyphen{}IP\textgreater{}:9090 aufrufen.

\end{itemize}

Auf der OLA Webseite prüfen:
\begin{itemize}
\item {} 
Ist OLA richtig konfiguriert?
Ist das DMX\sphinxhyphen{}Universum / sind die DMX\sphinxhyphen{}Universen richtig angelegt?

\item {} 
Ist ein OSC\sphinxhyphen{}Input in jedem Universum angelegt?

\item {} 
Sind die Outputs angelegt?

\end{itemize}

Um mithilfe der DMX Console Werte senden zu können, darf ClubDMX nicht laufen.
Sonst schicken sowohl DMX Console als auch ClubDMX Werte, das geht nicht. Sollte es
also mit der DMX Console möglich sein, DMX\sphinxhyphen{}Werte zu senden, dann kann man
davon ausgehen, dass ClubDMX nicht gestartet ist.
\begin{itemize}
\item {} 
ClubDMX stoppen.
Wenn der Alias \sphinxstyleemphasis{clubdmx} wie oben beschrieben angelegt ist,
dann kann im Terminal mit
dem Befehl \sphinxcode{\sphinxupquote{clubdmx stop}} ClubDMX angehalten werden. Wenn nicht, dann zuerst
ins clubdmx\_code Verzeichnis wechseln:

\begin{sphinxVerbatim}[commandchars=\\\{\}]
\PYG{n}{cd} \PYG{o}{/}\PYG{n}{home}\PYG{o}{/}\PYG{n}{pi}\PYG{o}{/}\PYG{n}{clubdmx\PYGZus{}code}
\PYG{o}{.}\PYG{o}{/}\PYG{n}{app\PYGZus{}start}\PYG{o}{.}\PYG{n}{sh} \PYG{n}{stop}
\end{sphinxVerbatim}

Nun kann mit
Sicherheit festgestellt werden, ob OLA mit der Hardware kommunizieren kann.
In der DMX Console müssen nun alle DMX Kanäle bedienbar sein und die
entsprechenden Geräte darauf reagieren.

\end{itemize}

\sphinxstylestrong{ClubDMX prüfen}

Wenn im vorigen Schritt ClubDMX gestoppt wurde, dann jetzt wieder starten mit
dem Befehl  \sphinxcode{\sphinxupquote{clubdmx start}}. Im Terminal werden nun nur wenige Meldungen
ausgegeben, die Hinweise auf das korrekte Funktionieren geben.
\begin{itemize}
\item {} 
Ist die Webseite erreichbar?

Im Browser die Seite \textless{}PI\sphinxhyphen{}IP\textgreater{}:5000 aufrufen.

Ist die Seite erreichbar? Erscheint im Browser eine Fehlermeldung?

\item {} 
Mehr Hinweise am Terminal erhalten

Im Normalfall wird ClubDMX als Daemon ausgeführt, das bedeutet, dass nur der
Start im Terminal festgestellt werden kann. Ab nun wird das Programm im
Hintergrund ausgeführt, Fehlermeldungen kommen also nicht aufs Terminal.

Gunicorn im Teminal und nicht als Daemon starten:

\begin{sphinxVerbatim}[commandchars=\\\{\}]
\PYG{n}{clubdmx} \PYG{n}{stop}
\PYG{n}{export} \PYG{n}{GUNICORNSTART}\PYG{o}{=}\PYG{l+s+s2}{\PYGZdq{}}\PYG{l+s+s2}{/home/pi/.local/bin/gunicorn}\PYG{l+s+s2}{\PYGZdq{}}
\PYG{n}{clubdmx} \PYG{n}{start}
\end{sphinxVerbatim}

Nun sieht man die Meldungen von ClubDMX auf dem Terminal. Das kann sicher
hilfreich bei der Fehlersuche sein.

Falls Gunicorn in einem anderen als dem oben genannten Verzeichnis
installiert wurde, dann muss zuerst der Pfad zu Gunicorn ermittelt werden.

\item {} 
Gunicorn finden

Gunicorn ist ein Python Package, das als Webserver für ClubDMX in Verwendung ist.
Packages können an unterschiedlichen Stellen installiert werden, was uns als User
grundsätzlich nicht kümmern muss. Allerdings sieht es etwas anders aus, wenn ein
Start\sphinxhyphen{}Skript einen exakten Pfad benötigt, weil zum Zeitpunkt des Startens noch kein
Suchpfad definiert ist.

Genau das ist beim Autostart von ClubDMX der Fall. Daher ist es nötig, dem Skript
app\_start.sh mitzuteilen, wo sich Gunicorn befindet. In der Regel wird sich
Gunicorn im Verzeichnis \sphinxstyleemphasis{/home/pi/.local/bin} befinden, aber das kann auch anders sein.

Mit dem Befehl

\sphinxcode{\sphinxupquote{find / \sphinxhyphen{}name "gunicorn*" 2\textgreater{}/dev/null}}

erhält man die Auskunft über den Speicherort von Gunicorn. Für den Fall, dass
ein anderer als der oben angegebene Pfad ermittelt wird, dann muss
die Environment\sphinxhyphen{}Variable \sphinxstyleemphasis{GUNICORNSTART} gesetzt werden (siehe
\sphinxstyleemphasis{Anmerkungen} weiter oben).

\end{itemize}

\begin{sphinxVerbatim}[commandchars=\\\{\}]
\PYG{n}{sudo} \PYG{n}{nano} \PYG{o}{/}\PYG{n}{etc}\PYG{o}{/}\PYG{n}{environment}
\PYG{c+c1}{\PYGZsh{} hier eintragen, Beispiel:}
\PYG{n}{export} \PYG{n}{GUNICORNSTART}\PYG{o}{=}\PYG{l+s+s2}{\PYGZdq{}}\PYG{l+s+s2}{/usr/bin/gunicorn3}\PYG{l+s+s2}{\PYGZdq{}}
\end{sphinxVerbatim}


\chapter{Entwicklungsschritte}
\label{\detokenize{history:entwicklungsschritte}}\label{\detokenize{history:history-label}}\label{\detokenize{history::doc}}
Die Idee zur Programmierung einer Lichtsteuerung trage ich schon lange Zeit
mit mir herum. Meine erste Software zur Lichtsteuerung stammt aus 1990\sphinxhyphen{}2000,
sie habe ich für meine Projekte als Lichttechniker auch gleich im Einstz
erproben können.

Hier gab es eine Software in C, die für eine Einzel\sphinxhyphen{}Anfertigung eines
Lichtpultes entwickelt wurde. Der Entwickler der Hardware Günter Humpel
lebt leider nicht mehr, ich möchte ihm hier an dieser Stelle meine
große Wertschätzung für seine Genialität ausdrücken.

Die Idee zu einer neuen Sotware schlummert seit dem Umstieg auf
kommerzielle Lichtpulte, diverse Versuche zur Realisierung verliefen
ohne Erfolg.

Als wesentlichen Zwischenschritt zur Entwicklung einer selbständigen
Software kann das Projekt \sphinxstylestrong{olaremote} bezeichnet werden, für das ich
erstmals Python und
\sphinxhref{https://www.openlighting.org/}{Open Lighting Architecture}
zur Fernsteuerung von Beleuchtung eingesetzt habe.

Der entscheidende Impuls zu \sphinxstylestrong{ClubDMX} kam von Paco Gonzalez\sphinxhyphen{}Rivero beim
Gespräch während einer Weihnachtsfeier. Paco überzeugte mich, dass
zeitgemäße Software ein Web\sphinxhyphen{}Interface zur Benutzerkontrolle haben muss.
Damit ist die Unabhängigkeit von Betriebssystemen und eine
breite Palette an Möglichkeiten für die Benutzer gegeben, ohne auf
spezifische, für ein Betriebssystem zugeschnittene Einzellösungen
gebunden zu sein.
So entstand die Idee einer simplen Lichtsteuerung mit Web\sphinxhyphen{}Interface
bei einem Smalltalk auf einer Weihnachtsfeier.


\section{Erste Idee}
\label{\detokenize{history:erste-idee}}
Die erste Idee war, Software\sphinxhyphen{}Fader für Lichtszenen zu programmieren.
Das ist die Grundidee: Lichtszenen in Kombination mit einer
Browser\sphinxhyphen{}Steuerung.

Dazu braucht es einiges an Vorbereitung.
\begin{itemize}
\item {} 
Wie werden die Daten gespeichert, Datenbank?

Ich habe mich für die Verwendung von CSV\sphinxhyphen{}Tabellen entschieden, um möglicht
unabhängig von Betriebssystem\sphinxhyphen{}Vorgaben zu sein. Hier war das Thema zu
lösen, wie gegebenenfalls Änderungen rückgängig gemacht werden können.

\item {} 
Klassische Lichttechnik arbeitet nach dem Prinzip der Kanal\sphinxhyphen{}Steuerung.
Zeitgemäße Lichttechnik hat mehr zu bieten, hier gibt es
unterschiedliche Möglichkeiten für unteschiedliche Geräte. Das
bedeutet einen erhöhten Level beim Einstieg in die Programmierung,
es braucht einen Patch, eine Unterscheidung zwischen HTP\sphinxhyphen{} und LTP\sphinxhyphen{}Attributen
und einige andere Konzepte mehr.

Ich wähle den spannenderen Weg und entscheide mich für die zeitgemäße
Umsetzung der Steuerung.

\item {} 
Welche Programmiersprache(n), welche Tools?

Hier sind einige Stichworte für meine Entscheidungen:
Visual Studio Code als Programmieroberfläche.
Python und Flask für die Grundlagen und die HTML\sphinxhyphen{}Seiten.
Jinja2 für die HTML\sphinxhyphen{}Templates.
Javascript und Jquery.
Bootstrap für das Design.

\end{itemize}


\section{Die Entwicklungsschritte}
\label{\detokenize{history:die-entwicklungsschritte}}\begin{itemize}
\item {} 
Version 0.1

Der Grundstein ist gelegt: Es gibt eine Patch als Kernstück. Hier
wird die Hardware in die Software eingebunden.
Und es gibt eine fixe Anzahl von 10 Lichtstimmungen, die über
Fader gesteuert werden können. Alle Daten zu Patch, Heads (= Geräte) und
Cues (= Lichtstimmungen) werden in CSV\sphinxhyphen{}Dateien geschrieben, da dies ein
Datenformat ist, das auf jedem Betriebssystem bearbeitbar ist.

Das Konzept der virtuellen Dimmer existiert bereits jetzt: Für LED Geräte
gibt es eine Trennung zwischen Intensität und Farbe.

\item {} 
Version 0.2

Die Anpassung an verschiedene Bedürfnisse wird wesentliches Thema.
Das heißt:
\begin{itemize}
\item {} 
eine erste Variante von Configurationsdatei entsteht.

\item {} 
Die Anzeige der CSV\sphinxhyphen{}Tabellen und deren Editierbarkeit werden
überarbeitet.

\item {} 
Die Web\sphinxhyphen{}Funktionen befinden sich allesamt in einer immer länger
werdenden Datei.

\end{itemize}

\item {} 
Version 0.3

Die Datenbankstruktur wird einheitlich und damit universeller. Jede
Kategorie an Daten erhät ein eigenes Verzeichnis. Die Anzahl der Cues ist
konfigurierar und nicht mehr auf 10 beschränkt. Damit ist der Grundstein
für eine modulare Struktur gelegt, allerdings wird die Datei mit den
Web\sphinxhyphen{}Funktionen zu einem Monster, weil mir die Infos zur
Strukturierung fehlen. In dieser Programmstruktur ist das Limit erreicht.

Messages: Die Rückmeldungen der Software werden überarbeitet und
einheitlicher.

\item {} 
Version 0.4

Sehr wichtige Neuerungen:
\begin{itemize}
\item {} 
Die Stage

Damit ist eine erste Geräte Einzelsteuerung geschaffen. Auf der
Stage\sphinxhyphen{}Fläche können Symbole, die für Geräte (Heads) oder Text stehen,
platziert werden.

\item {} 
Neue Programmstruktur: Die Webseiten werden in Module aufgeteilt.

Für die Benutzer zwar nicht relevant, aber mit dieser Struktur werden
viele Überschneidungen und Nebeneffekte vermieden.

\item {} 
Neues Datenkonzept: Alle Daten werden in einem \sphinxstylestrong{Projekt}
zusammengefasst. Diese Zusammenfassung ermöglicht das Wechseln in
verschiedene Einheiten von Datenbanken.

\end{itemize}

\item {} 
Version 0.5

Aus dem \sphinxstylestrong{Projekt} wird der \sphinxstylestrong{Raum}. Dieser Begriff ist verständlicher
und beschreibt besser, worauf die Datenstruktur abzielt. In einem Raum
können mehrere \sphinxstylestrong{Konfigurationen} abgespeichert werden, die sich die
Datenbanken teilen, wie Cues, Stage, Patch etc.

Der \sphinxstylestrong{Raum} wird mit Environment\sphinxhyphen{}Vaiable vor Start der App bestimmt.
Ein Wechsel des Raums ist noch nicht implementiert.

Für die Tabellen gibt es erweiterte Editierfunktionen: Zelleneditor und
Zeilenselektion. Mittels Selektion können Zeilen mit copy/paste
ausgeschnitten und vervielfältigt werden.

\item {} 
Version 0.6

Die in dieser Version wichtigste Neuerung ist das Login, damit sind die
Webseiten und die Steuerung vor unbefugtem Zugriff geschützt.

Eine weitere Neuerung ist die Midi\sphinxhyphen{}Steuerung. Die Cues können mit einem
Midi\sphinxhyphen{}Faderboard (Korg nanoKontrol 1 oder 2) geregelt werden.

Neben der Stage\sphinxhyphen{}Seite gibt es nun auch eine Stage\sphinxhyphen{}Kompakt Seite, die
für die Verwendung von kleinen Touch\sphinxhyphen{}Bildschirmen konzipiert ist. Da
Selektion und Verschieben nicht verträglich mit Seite\sphinxhyphen{}Verschieben ist,
braucht es diese Möglichkeit.

Neu hinzu kommen auch Buttons: Mit Buttons können Cues zeitabhängig
ein\sphinxhyphen{} und ausgefadet werden.

\item {} 
Version 0.7

Ab nun ist eine erweiterte Raum\sphinxhyphen{}Bedienung möglich: Das Wechseln des
Raums, Backup und Restore auf USB\sphinxhyphen{}Stick. Diese USB\sphinxhyphen{}Funktionen arbeiten
plattformübergreifend auf Windows und Linux.

Das Editieren der Konfiguration wird wesentlich erleichtert: Es geht nun
in einem Formular, wo für jedes Feld die entsprechenden Optionen
ausgewählt werden können.

Eine neue Stage\sphinxhyphen{}Ansicht kommt hinzu: Der Single\sphinxhyphen{}Modus. Hier können die
Attribute aller Geräte einfach bedient werden, die geänderten
Attribute sind im {\hyperref[\detokenize{grundlagen:topcuelabel}]{\sphinxcrossref{\DUrole{std,std-ref}{Topcue}}}} und können so abgespeichert werden.

\item {} 
Version 0.8

Hier kommen wieder einige neue Funktionen dazu.
\begin{itemize}
\item {} 
Es gibt eine Demo\sphinxhyphen{}Version auf \textless{}\sphinxurl{https://guntherseiser.pythonanywhere.com}\textgreater{}

\item {} 
Änderungen in Fadern, Buttons und der Stage werden auf allen
offenen Browser\sphinxhyphen{}Fenstern aktualisiert.

\item {} 
Es gibt eine neue Seite, die den DMX\sphinxhyphen{}Output anzeigt und regelmäßig
aktualisiert

\item {} 
Es gibt eine eigene Seite für die Dokumentation.

\item {} 
Die Benutzer\sphinxhyphen{}Rollen werden detailliert unterschieden. Somit haben die
Benutzer entsprechend ihrer Rolle unterschiedliche Möglichkeiten
zur Interaktion und bekommen auch unterschiedliche Webseiten zur
Ansicht. Siehe {\hyperref[\detokenize{benutzer:benutzer-label}]{\sphinxcrossref{\DUrole{std,std-ref}{Benutzer}}}} .

\item {} 
Die Seite zum Benutzer\sphinxhyphen{}Management wird überarbeitet.

\item {} 
Das Speichern des Topcue wird vereinfacht. Nun kann der Topcue
als Fader oder Button abgespeichert werden.

\item {} 
Die Raum\sphinxhyphen{}Sicherung wird erweitert um upload/download.

\end{itemize}

\item {} 
Version 0.9

Diverse Vereinfachungen beim Benutzen, zum Beispiel Auswahlfelder und
Inhalts\sphinxhyphen{}Überprüfung beim Editieren der Felder in der
Datenbank.

Neue Funktionen:
\begin{itemize}
\item {} 
Buttons mit verschiedenen Aufgaben: Schalter, Taster, Auswahlschalter

\item {} 
OSC Input

\item {} 
In den Seiten Stage, Fader und Button: Einrichtungs\sphinxhyphen{}Symbol für
Sprung zur jeweiligen Setup\sphinxhyphen{}Tabelle.

\item {} 
Raum\sphinxhyphen{}Management: umbenennen, unter anderem Namen sichern, löschen.

\item {} 
Die Exec\sphinxhyphen{}Seite beinhaltet Buttons und Fader. Ist die Startseite für den
Basic\sphinxhyphen{}Benutzer.

\item {} 
Fader und Buttons müssen nicht zwingend auf der Exec\sphinxhyphen{}Seite zu sehen sein,
sie können auch auf zusätzlichen Seiten platziert werden.

\item {} 
Start\sphinxhyphen{}Cue: Es kann ein Cue konfiguriert werden, der beim Start von
ClubDMX ausgeführt wird.

\item {} 
Patch\sphinxhyphen{}Formular zum Erzeugen mehrerer Heads in einem Schritt.
siehe {\hyperref[\detokenize{patch:neupatchlabel}]{\sphinxcrossref{\DUrole{std,std-ref}{Neuen Patch erstellen}}}} .

\item {} 
Kommandozeilen\sphinxhyphen{}Skripte \sphinxcode{\sphinxupquote{sendosc.sh}} und \sphinxcode{\sphinxupquote{sendosc.bat}} zum Senden von
OSC\sphinxhyphen{}Befehlen, siehe {\hyperref[\detokenize{einrichten:crontab-label}]{\sphinxcrossref{\DUrole{std,std-ref}{Einrichten von zeitabhängiger Steuerung mit Crontab (in Linux)}}}}.

\item {} 
Stage: Geräte\sphinxhyphen{}Fader werden in einem Fenster angezeigt, Selektion kann gemeinsam
verschoben werden.

\item {} 
Stage: In der Config (Bedienelemente) wird eine Default\sphinxhyphen{}Stage angegeben. Die
angezeigte Stage kann davon abweichen: Die angezeigte Stage wird in einer
Session\sphinxhyphen{}Variablen gespeichert.

\end{itemize}

\end{itemize}


\chapter{Impressum}
\label{\detokenize{impressum:impressum}}\label{\detokenize{impressum::doc}}
Gunther Seiser
Mascagnigasse 27
5020 Salzburg

Telefon: +43 699 8188 1989

\sphinxhref{mailto:gunther.seiser.63@gmail.com}{gunther.seiser.63@gmail.com}

Erstellt  am 28.01.2022



\renewcommand{\indexname}{Stichwortverzeichnis}
\printindex
\end{document}